\documentclass[]{elsarticle} %review=doublespace preprint=single 5p=2 column
%%% Begin My package additions %%%%%%%%%%%%%%%%%%%
\usepackage[hyphens]{url}

  \journal{Transportation Research Part D: Transport and Environment} % Sets Journal name


\usepackage{lineno} % add
\providecommand{\tightlist}{%
  \setlength{\itemsep}{0pt}\setlength{\parskip}{0pt}}

\usepackage{graphicx}
\usepackage{booktabs} % book-quality tables
%%%%%%%%%%%%%%%% end my additions to header

\usepackage[T1]{fontenc}
\usepackage{lmodern}
\usepackage{amssymb,amsmath}
\usepackage{ifxetex,ifluatex}
\usepackage{fixltx2e} % provides \textsubscript
% use upquote if available, for straight quotes in verbatim environments
\IfFileExists{upquote.sty}{\usepackage{upquote}}{}
\ifnum 0\ifxetex 1\fi\ifluatex 1\fi=0 % if pdftex
  \usepackage[utf8]{inputenc}
\else % if luatex or xelatex
  \usepackage{fontspec}
  \ifxetex
    \usepackage{xltxtra,xunicode}
  \fi
  \defaultfontfeatures{Mapping=tex-text,Scale=MatchLowercase}
  \newcommand{\euro}{€}
\fi
% use microtype if available
\IfFileExists{microtype.sty}{\usepackage{microtype}}{}
\bibliographystyle{elsarticle-harv}
\ifxetex
  \usepackage[setpagesize=false, % page size defined by xetex
              unicode=false, % unicode breaks when used with xetex
              xetex]{hyperref}
\else
  \usepackage[unicode=true]{hyperref}
\fi
\hypersetup{breaklinks=true,
            bookmarks=true,
            pdfauthor={},
            pdftitle={Examining spatial equity and accessibility to a public bicycle share program using a balanced floating catchment area approach},
            colorlinks=false,
            urlcolor=blue,
            linkcolor=magenta,
            pdfborder={0 0 0}}
\urlstyle{same}  % don't use monospace font for urls

\setcounter{secnumdepth}{0}
% Pandoc toggle for numbering sections (defaults to be off)
\setcounter{secnumdepth}{0}

% Pandoc citation processing

% Pandoc header
\usepackage{booktabs}
\usepackage{longtable}
\usepackage{array}
\usepackage{multirow}
\usepackage{wrapfig}
\usepackage{float}
\usepackage{colortbl}
\usepackage{pdflscape}
\usepackage{tabu}
\usepackage{threeparttable}
\usepackage{threeparttablex}
\usepackage[normalem]{ulem}
\usepackage{makecell}
\usepackage{xcolor}



\begin{document}
\begin{frontmatter}

  \title{Examining spatial equity and accessibility to a public bicycle share
program using a balanced floating catchment area approach}
    \author[Some School]{Elise Desjardins\corref{1}}
   \ead{desjae@mcmaster.ca} 
    \author[Another University]{Christopher D. Higgins\corref{2}}
   \ead{cd.higgins@utoronto.ca} 
    \author[Some School]{Antonio Paez\corref{2}}
   \ead{paezha@mcmaster.ca} 
      \address[McMaster University]{School of Earth, Environment \& Society, 1280 Main Street West,
Hamilton, ON L8S4L8}
    \address[University of Toronto Scarborough]{Department of Geography \& Planning, 1265 Military Trail, Toronto, ON
M1C1A4}
      \cortext[1]{Corresponding Author}
    \cortext[2]{Equal contribution}
  
  \begin{abstract}
  Public bicycle share programs are implemented to promote cycling as a
  convenient and sustainable mode of transportation. These systems can
  also play a role in addressing transportation needs and advancing
  transportation equity if they make cycling more accessible to lower
  income or disadvantaged populations. Recent studies assessing equity in
  bike sharing have found differences in ridership or membership based on
  income, education, and trip purpose. In Ontario, the City of Hamilton
  launched a public bicycle share program in 2015 that currently has over
  900 operational bicycles and 130 docking stations. In 2018, the
  non-profit organization that operates the program launched an equity
  initiative to provide subsidized memberships and to expand service by
  adding twelve docking stations. Previous research found that Hamilton's
  public bicycle share program targeted well disadvantaged areas in the
  city compared to programs in other Canadian cities. Since the time or
  distance that members of the population need to reach a bicycle share
  station decreases the potential of accessing the program, the location
  of stations matters. Unlike other public amenities with greater ability
  to adapt to crowding, the utility of stations is limited by the number
  of bicycles that they can hold, which makes crowding effects critical;
  the system may not necessarily be improved if stations are easy to reach
  but offer only a small number of bicycles. In this research, we revisit
  the case of Hamilton and investigate equity differentials in
  accessibility to bicycle share stations. We compare accessibility in the
  program with and without the equity stations to assess the effect of the
  initiative. Previous research on cycling equity or accessibility has
  used a macro-level approach with entire neighbourhoods or census tracts
  as the unit of analysis, although there are recent exceptions. In
  contrast, we implement our analysis parting from microscopic zones to
  better reflect travel to a docking station, which often happens at the
  sub-neighborhood level, and conduct a sensitivity analysis at several
  walking thresholds. We then reaggregate the estimated accessibility to
  dissemination areas for further analysis using census data. Equity
  analysis of this type is possible thanks to a newly developed approach
  for accessibility using balanced floating catchment areas. The addition
  of equity stations achieved its goal to increase accessibility for lower
  income populations, although the increase was relatively modest (2.10
  bicycles/person with equity stations vs 1.97 bicycles/person without
  equity stations). Further research is needed to determine whether this
  encouraged more cycling.
  \end{abstract}
  
 \end{frontmatter}

\hypertarget{background}{%
\section{Background}\label{background}}

Public bicycle share programs (PBSP) have been implemented in over 800
cities worldwide and a great deal has been learned about their typical
users ({\textbf{???}}). In many cities, males use bike share more than
females ({\textbf{???}}; {\textbf{???}}; {\textbf{???}}; {\textbf{???}})
as do younger age cohorts ({\textbf{???}}; {\textbf{???}};
{\textbf{???}}). However, one study found that bike share users in
Washington, DC were more likely to be female ({\textbf{???}}), which
suggests that the gender gap among cyclists who use bike share is less
disparate than the gap for personal bicycle use ({\textbf{???}}). There
is some evidence that bike share users are less likely to own a car
({\textbf{???}}; {\textbf{???}}). However, the relationship between
income or education and bike share use is less clear. Stations in
disadvantaged communities in Chicago generated most of the average
annual trips ({\textbf{???}}) and individuals from minority or lower
socioeconomic status neighbourhoods in Minneapolis-St.~Paul used the
city's PBSP more ({\textbf{???}}). Being university educated was also a
significant correlate of bike share use in Montreal, Canada
({\textbf{???}}). On the other hand, members who reside in
minority-concentrated and lower socioeconomic status neighbourhoods use
the Minneapolis-St Paul bike share more frequently ({\textbf{???}}).
Financial savings have been found to motivate those on a low income to
use bike share ({\textbf{???}}). Many studies have found that proximity
to a bike share station, either living or working near one, is an
important determinant of use or having a membership ({\textbf{???}};
{\textbf{???}}). This makes sense given that individuals are more likely
to use services or programs that they can easily access. Finally, the
potential of bike share programs to increase cycling levels has also
been explored recently ({\textbf{???}}; {\textbf{???}}), but more
longitudinal evidence is needed to determine their impact on encouraging
more cycling and offering more opportunities for physical activity. On
the whole, these findings highlight the need for bike share systems to
be more highly accessible for diverse populations in order to increase
use beyond the ``typical'' users, which has been the focus of recent
research ({\textbf{???}}; {\textbf{???}}; {\textbf{???}}), and to offer
more people the option of using sustainable and active transportation.

Several studies have recently explored the equity of PBSP in North
American cities by primarily examining who has access to bike share
(e.g., differences by demographics or socioeconomic status) and where
stations are located. Equity can be achieved in two different ways:
equal spatial distribution across a region (e.g., horizontal equity) or
greater access for vulnerable or disadvantaged populations (e.g.,
vertical equity) ({\textbf{???}}). Both are of interest to researchers
and transport planners since they are often linked in that advantage, or
conversely disadvantage, has spatial patterns. Using a negative binomial
regression model, Qian and Jaller ({\textbf{???}}) estimated ridership
in Chicago's PBSP among disadvantaged communities and found some
disparities. While annual members in disadvantaged communities have a
significantly lower share of trips compared to other areas in the city,
they make longer trips. This suggests that they may be using PBSP for
work commuting, which points to the importance of ensuring equitable
access ({\textbf{???}}). Similar results were found in Philadelphia,
where lower income areas generate fewer trips and efforts to increase
equity within the program have not been as successful as intended
({\textbf{???}}). In the case of Seattle, all neighbourhoods had some
level of access to dockless bikes but those with higher incomes and more
residents of higher education had more bikes ({\textbf{???}}). Babagoli
et al.~({\textbf{???}}) found that neighbourhoods in New York City with
higher affluence had the greater proportion of Citi Bike stations.
Overall, these findings suggest that horizontal equity can be achieved
while vertical equity is harder to attain. Exploring transport equity by
investigating where bicycle share stations are located, often using
neighbourhoods or census tracts as the spatial unit of analysis, can
ignore or miss the benefits that may be derived from adjacent zones.
Meaning that, stations may be lacking in certain neighbourhoods but
there may be stations accessible within a reasonable walking cutoff
time. This is where accessibility becomes an important consideration.

Accessibility has been applied in both a positive and normative way to
inform transportation planning ({\textbf{???}}), but its utility to this
field has evolved since its conceptualization and become linked with
recent interest in prioritizing local proximity and modes that are
suitable for local trips like walking and cycling ({\textbf{???}}). As a
measure of the ease of reaching potential destinations spread spatially
in a given area, accessibility is relevant to PBSPs because it can
identify current inequities in usage, as well as guide interventions
that increase access for groups that are under-serviced or address gaps
in transportation options. It also addresses some of the challenges of
other performance measures such as level of service within a
transportation network by measuring person-based indicators and
exploring differences in use between population groups ({\textbf{???}}).

In the context of PBSPs, accessibility refers to the distance an
individual must travel to reach a bicycle share station ({\textbf{???}};
{\textbf{???}}). Since the time or distance that members of the
population need to reach a bicycle share station decreases the potential
of accessing the program and ultimately use of the program, the location
and size (e.g., maximum number of bicycles available) of stations
matters. Indeed, distance to bicycle share stations is associated with
use ({\textbf{???}}; {\textbf{???}}) and can be a barrier to using PBSPs
({\textbf{???}}). Kabra et al.~({\textbf{???}}) found that the majority
of bike share usage in Paris comes from areas within 300m of stations.
Furthermore, unlike other public amenities with greater ability to adapt
to crowding, like health care services, the utility of stations is
limited by the number of bicycles that they can hold, which makes
crowding effects critical. The program may not necessarily be improved
if stations are easy to reach but offer only a small number of bicycles.
Likewise, more people may not opt to use the program if the supply of
bicycles available at the nearest station is insufficient to meet
demand. The location and size of stations is important to increase the
utility of this public transportation option for more people, thus
achieving vertical equity. Accessibility analyses for PBSPs constitute a
positive and evaluation-based approach that also has the potential to
inform equity efforts. For instance, Wang and Lindsey ({\textbf{???}})
investigated whether new or relocated bicycle share stations increased
accessibility and use, which offered important insights to improve the
performance of the program.

Several approaches have been commonly used for measuring accessibility:
cumulative opportunities, gravity, and utility-based ({\textbf{???}}).
Paez et al ({\textbf{???}}) provide a recent overview of various
formulations and applications of accessibility in transportation
research. Briefly, the gravity-based approach involves weighting
opportunities, for example the quantity of bicycle share stations, as
measured by a function of time, using a negative exponential form
({\textbf{???}}). It takes both demand and congestion into account
({\textbf{???}}). Floating catchment area (FCA) methods are a type of
gravity measure that have been used in healthcare accessibility research
but that are applicable to transportation case studies. This approach
involves producing flexible catchment areas for populations, recognizing
that people may be willing to travel to access particular services
({\textbf{???}}). Thus, this model does a good job of considering
potential crowding of services using a binary distance function
({\textbf{???}}). A recent improvement to this approach was achieved
through a simple and intuitive adjustment to the allocation of supply
and demand, which addressed the effects of demand and service inflation
from the conventional approach (see {\textbf{???}}).

Researchers have also taken different approaches when it comes to
aggregation of data, either by using the individual or household as the
smallest unit of analysis or larger spatial zones. Previous research on
bike share equity has typically used a macro-level approach with
aggregate data from entire neighbourhoods or census tracts
({\textbf{???}}; {\textbf{???}}; {\textbf{???}}; {\textbf{???}}),
although there are recent exceptions ({\textbf{???}}; {\textbf{???}}).
This is also true for studies examining correlates of bike share demand
({\textbf{???}}). Handy and Niemeier ({\textbf{???}}) note that using
disaggregated data in accessibility analyses provides a more accurate
estimate for individuals, which is useful for addressing vertical
inequities in PBSP usage.

Using balanced floating catchment area methods, a novel approach that
has not been used yet in cycling research, we examine accessibility to a
public bicycle share program in Hamilton, Ontario. Using disaggregate
population-level data, we (1) conduct a sensitivity analysis by
measuring accessibility and level of service to bike share stations at
different thresholds of walking to reach a station: 3 minutes, 5
minutes, 10 minutes, and 15 minutes; (2) explore the contribution of
specific stations that were added to Hamilton's PBSP to reducing both
horizontal and vertical inequities; and (3) examine whether disparities
in accessibility exist according to median household income of
dissemination areas within the core service area.

\hypertarget{sec:study}{%
\section{Case Study}\label{sec:study}}

This paper uses the city of Hamilton, located in Ontario, Canada, as a
case study. The city launched a public bicycle share program, Hamilton
Bike Share, in March 2015 with 115 stations and 750 bicycles
({\textbf{???}}). Before June 2020, the program was known as SoBi
Hamilton. Stations are spaced between 300 and 600 metres apart
({\textbf{???}}). The core service area spans 35 km\^{}2 of the city and
roughly 138,000 can reach a bike share hub within 30 minutes of walking
{[}see Figure 1{]}. This represents roughly one fifth of the population
in Hamilton, according to the 2016 Canadian Census. The program was
enthusiastically welcomed in the city - within three weeks of launching,
10,000 trips had been made ({\textbf{???}}). In 2017, Hamilton Bike
Share Inc., the non-profit organization that operates the program,
initiated an equity program, Everyone Rides Initiative (ERI), to remove
barriers that may prevent individuals from accessing bike share in
Hamilton. An additional 75 bicycles and 15 stations were added to the
program, which expanded it to more disadvantaged areas in the core
service area {[}see Figure 2{]}. The program also offers subsidized
memberships to individuals who identify as low income. A comparable
program can be found in Philadelphia (see {\textbf{???}}). As of June
2020, the bike share program has 900 bikes and 130 stations, and over
26,000 active memberships ({\textbf{???}}).

Hamilton Bike Share has conducted one membership survey to date in 2018
({\textbf{???}}), and the findings from 420 members are broadly in line
with the trends that were discussed above (see {\textbf{???}} for a
recent review of the literature). The majority of respondents live
within the core service area and the gender split is typical: 57\% of
respondents are male and 41\% are female. The majority of respondents,
both male and female, are between 25 and 34 years of age, but the
percentage of male respondents is higher in the subsequent age groups.
Respondents use bike share for commuting (40\% of trips) or errands and
meetings (24\% of trips), and nearly 50\% of trips have an average
length of 11 to 20 minutes. As a result of having a bike share
membership, 49\% of respondents report that they use their private
vehicle less often or much less often but 48\% report that their private
vehicle use has remained about the same. This suggests that SoBi
Hamilton has been useful for certain kinds of trips but not all, meaning
that some trips continue to require a private vehicle.

Further research has provided insights about the behaviour and route
choice of bike share users in Hamilton. The routes most frequently
travelled are longer than the shortest distance route from origin to
destination ({\textbf{???}}; {\textbf{???}}). Ridership is influenced by
weather conditions, temporal factors such as university terms,
employment, and proximity to important destinations like the local
university, but is not influenced by population and most transportation
infrastructure variables ({\textbf{???}}). Bike share users prefer to
deter from the shortest distance route to travel on streets with cycling
facilities or lower volumes of traffic ({\textbf{???}}) and avoid steep
slopes and busy roads ({\textbf{???}}). It is important to note that
these studies used daily ridership or GPS data from trips taken before
the equity stations were implemented, so these findings are informative
with respect to the conventional stations.

Our analysis builds upon a previous and recent study ({\textbf{???}}),
which found that disadvantaged areas in Hamilton are better served by
the city's PBSP compared to other Canadian cities {[}i.e., Toronto,
Vancouver, Montreal, and Ottawa-Gatineau{]} with PBSPs where advantaged
areas have greater access. Hosford and Winters ({\textbf{???}})
acknowledge that ``Hamilton stands out in that the lower income
neighborhoods are located near the city center and wealthier
neighborhoods are in the surrounding suburban areas''. Therefore, the
core service area for the PBSP in Hamilton is by default in more
disadvantaged areas compared to other Canadian cities, but there is also
a great deal of variation in income in the city center as well because
of the local university and increasing gentrification in the downtown
area. In their analysis, Hosford and Winters ({\textbf{???}}) did not
differentiate between the conventional stations and those added to
increase equity, which we refer to as equity stations in this paper.
Therefore, this paper continues their work and makes an important
contribution to the growing body of literature that examines equity
issues in PBSPs.

\begin{figure}

{\centering \includegraphics[width=0.65\linewidth]{Bike-share-spatial-equity_files/figure-latex/hamilton-and-sobi-service-area-1} 

}

\caption{The core service area of Hamilton Bike Share is outlined in blue. Dissemination areas in the Hamilton Census Metropolitan Area (CMA) are outlined in grey.}\label{fig:hamilton-and-sobi-service-area}
\end{figure}

\begin{figure}

{\centering \includegraphics[width=1\linewidth]{Bike-share-spatial-equity_files/figure-latex/sobi-stations-in-hamilton-1} 

}

\caption{The spatial distribution of bike share stations in Hamilton, Ontario. The service area of Hamilton Bike Share is outlined in blue and the city's downtown core is outlined in light blue. Dissemination areas within the Hamilton CMA that touch the bounding box of the core service area are outlined in grey.}\label{fig:sobi-stations-in-hamilton}
\end{figure}

\hypertarget{sec:methods}{%
\section{Methods}\label{sec:methods}}

\hypertarget{floating-catchment-area}{%
\subsection{Floating Catchment Area}\label{floating-catchment-area}}

Floating catchment area (FCA) methods are an approach commonly used in
the healthcare accessibility literature. This approach is more
appropriate and informative than calculating the provider-to-population
ratio (PPR) which simply divides the level of supply of a service (e.g.,
the number of bicycle racks at a station) by the population who have
access to the service ({\textbf{???}}). In particular, the Two-Step
Floating Catchment Area method produces flexible catchment areas for
populations accessing a service rather than using rigid boundaries like
PPR, which provides more useful information because it takes into
account individual behaviour and doesn't assume that people stay within
the pre-defined boundaries ({\textbf{???}}). This is an important
property that supports our rationale for applying this method to measure
accessibility to Hamilton Bike Share. The City of Hamilton has
positioned stations between 300 and 600 metres apart, which would lead
to boundaries around these buffers, but it would be reasonable to assume
that people are willing to walk beyond this threshold to access other
stations if the ones nearest them have no supply of bicycles.

More recently, the \emph{balanced} floating catchment area approach was
developed to address issues with demand and supply inflation that are a
result of the overlapping catchment areas produced by the Two-Step
Floating Catchment Area ({\textbf{???}}). This overlap inflates the
demand and generates inaccurate or misleading accessibility estimates
({\textbf{???}}). By adjusting the impedance weights, both supply and
demand are proportionally allocated which eliminates the double-counting
of population that leads to demand and supply inflation
({\textbf{???}}). Other benefits of this adjusted method include
consideration of competition which can occur when catchment areas
overlap, as well as the preservation of system-wide population and level
of service. Balanced floating catchment area methods have been used to
explore accessibility to health care providers ({\textbf{???}}) and
COVID-19 health care services ({\textbf{???}}), but have not yet been
used in the cycling literature to explore issues of accessibility.

In their review of accessibility measures, Geurs and van Wee
({\textbf{???}}) highlight the need for greater inclusion of individual
spatio-temporal constraints but acknowledge the challenges of acquiring
and analyzing person-based data. This comes after Kwan's
({\textbf{???}}) work to show that space-time measures are more capable
of capturing interpersonal differences, especially the effect of
space-time constraints on individual behaviour, and are more helpful for
unraveling gender/ethnic differences. Applying the balanced floating
catchment area approach to examine accessibility to Hamilton Bike Share
constitutes a location-based measure with an individual component by
stratifying according to median household income ({\textbf{???}}), but
conducting a further sensitivity analysis by adjusting distance
thresholds would introduce additional spatio-temporal constraints to
evaluate equitable accessibility.

The first step in the balanced floating catchment area method is to
allocate the population to be serviced by each Hamilton Bike Share
station: \[
P_j = {\sum_{i = 1}^{n} P_i{w_{ij}}}
\]

Next, the level of service at each station (i.e., the maximum number of
bicycle racks) is divided by its estimated service population within the
established catchment area: \[
L_j = \frac {S_j}{P_j} = \frac {S_j}{{\sum_{i = 1}^{n} P_i{w_{ij}}}}
\]

Finally, the accessibility of population cell \(i\) is calculated as the
weighted sum of the level of service of all stations that can be reached
from there according to normalized weights: \[
A_i = {\sum_{j = 1}^{J} L_j{w_{ji}}} = {\sum_{j = 1}^{J} \frac {S_j{w_{ji}}}{\sum_{i = 1}^{n} P_i{w_{ij}}}}
\]

The approach uses instead a set of suitably normalized weights as
follows: \[
{w_{ij}^{i} = \frac {w_{ij}}{\sum_{j = 1}^{J} {w_{ji}}}}
\]

and \[
{w_{ij}^{j} = \frac {w_{ij}}{\sum_{j = 1}^{J} {w_{ji}}}}
\]

These weights satisfy the following properties: \[
\sum_{j = 1}^{J} {w^i_{ji}} = 1
\]

and \[
\sum_{i = 1}^{n} {w^j_{ji}} = 1
\]

Finally, accessibility is calculated without risk of demand or supply
inflation: \[
A_i = {\sum_{j = 1}^{J} \frac {S_j{w^j_{ij}}}{\sum_{i = 1}^{n} P_i{w^i_{ij}}}}
\]

\hypertarget{pycnophylactic-interpolation}{%
\subsection{Pycnophylactic
Interpolation}\label{pycnophylactic-interpolation}}

We first plotted the population in all dissemination areas within SoBi
Hamilton's core service area. Since pycnophylactic interpolation occurs
at such a micro scale, we had to ensure that population numbers were not
interpolated or allocated to areas where people do not live in Hamilton
(i.e., where schools or parks are located). To do so, we retrieved
shapefiles for various geographic features (see Table 1) from Open
Hamilton. Figure 3 shows where these spaces are located within SoBi
Hamilton's core service area. Next, we extracted these features from the
PBSP core service area and used pycnophylactic interpolation to
disaggregate and reallocate population from dissemination areas to
smaller polygons. These micro-level polygons are 50 by 50 metres.

\begin{table}

\caption{\label{tab:data-features}\label{tab:landscape-features}Landscape features}
\centering
\resizebox{\linewidth}{!}{
\begin{tabular}[t]{>{}l|>{\raggedright\arraybackslash}p{30em}}
\toprule
Feature & Description\\
\midrule
\cellcolor{gray!6}{\textbf{Hamilton Bike Share Stations}} & \cellcolor{gray!6}{The location of stations and the number of racks available at each station.}\\
\textbf{Golf Courses} & The location of City and privately owned golf courses.\\
\cellcolor{gray!6}{\textbf{Parks}} & \cellcolor{gray!6}{The location of parks and other green spaces.}\\
\textbf{Designated Large Employment Areas} & The location of large business parks and industrial lands.\\
\cellcolor{gray!6}{\textbf{Municipal Parking Lots}} & \cellcolor{gray!6}{The location of municipal car parks.}\\
\addlinespace
\textbf{Cemeteries} & The location of cemeteries.\\
\cellcolor{gray!6}{\textbf{Environmentally Sensitive Areas}} & \cellcolor{gray!6}{The location of either land or water areas containing natural features or significant ecological functions.}\\
\textbf{Streets} & The street network in Hamilton, including road classification for highways.\\
\cellcolor{gray!6}{\textbf{Educational Institutions}} & \cellcolor{gray!6}{The location of all educational institutions and schools.}\\
\textbf{Places of Worship} & The location of buildings used for religious congregations.\\
\addlinespace
\cellcolor{gray!6}{\textbf{Municipal Service Centres}} & \cellcolor{gray!6}{The location of all municipal service centres, including City Hall.}\\
\textbf{Recreation and Community Centres} & The location of all recreation and community centres.\\
\cellcolor{gray!6}{\textbf{Arenas}} & \cellcolor{gray!6}{The location of all indoor arenas.}\\
\textbf{Emergency Stations} & The location of all Emergency Management Services (EMS) Ambulance stations.\\
\cellcolor{gray!6}{\textbf{Fire Stations}} & \cellcolor{gray!6}{The location of all fire stations.}\\
\addlinespace
\textbf{Police Stations} & The location of all police stations.\\
\cellcolor{gray!6}{\textbf{Railways}} & \cellcolor{gray!6}{The railway network in Hamilton.}\\
\textbf{Hospitals} & The location of all hospitals.\\
\bottomrule
\end{tabular}}
\end{table}

\hypertarget{travel-time-matrix}{%
\subsection{Travel Time Matrix}\label{travel-time-matrix}}

\href{https://download.bbbike.org/osm/bbbike/}{BBBike} is an online
cycle route planner that interfaces with OpenStreetMap. We extracted
OpenStreetMap data for SoBi Hamilton's service area to calculate walking
times from each population cell to nearby bike share stations, using a
walking distance of 10km and a walking time of 30 minutes as a
threshold. A travel time matrix was created with the origins as the
coordinates of the population cells and the destinations as the
coordinates of the bike share stations within the maximum threshold.

\hypertarget{data}{%
\subsection{Data}\label{data}}

All data for this research were accessed from Open Hamilton\footnote{https://open.hamilton.ca/},
an online repository of data curated by the City of Hamilton.

\hypertarget{results}{%
\section{Results}\label{results}}

\begin{figure}

{\centering \includegraphics[width=0.65\linewidth]{Bike-share-spatial-equity_files/figure-latex/figure-2-1} 

}

\caption{Population in all dissemination areas in the SoBi Hamilton core service area and within 30 minutes of walking to a bike share station.}\label{fig:figure-2}
\end{figure}

\begin{figure}

{\centering \includegraphics[width=0.65\linewidth]{Bike-share-spatial-equity_files/figure-latex/figure-4-1} 

}

\caption{Interpolated population cells in the SoBi Hamilton core service area (outlined in black) and within 30 minutes of walking to the core service area (outlined in blue). The downtown area is outlined in dark green. Geographic features have been extracted. Dissemination area polygons are outlined in grey.}\label{fig:figure-4}
\end{figure}

\hypertarget{accessibility-by-distance-thresholds}{%
\subsection{Accessibility by Distance
Thresholds}\label{accessibility-by-distance-thresholds}}

Consensus regarding the distance that individuals are willing to walk to
access a PBSP is lacking, therefore the literature on walking behaviour
was consulted to determine the thresholds for the sensitivity analysis.
Previous studies have found that living within 250 metres
({\textbf{???}}) and 300 metres ({\textbf{???}}) is correlated with bike
share use, while other research has found that walking trips are less
than 600 metres and rarely more than 1200 metres ({\textbf{???}}) or a
median distance of 650 metres ({\textbf{???}}). As seen in Table
\ref{tab:accessibility-sensibility}, we find that accessibility
increases with a threshold between two and four minutes, but is then
maximized at 5 minutes. At some stations, Hamilton Bike Share will
depict a map to show the user the locations of the other nearest
stations within a five minute walk, which suggests that this is an
average distance that people are anticipated to walk. Accessibility
decreases significantly after eight minutes, which is intuitive given
that demand on a limited supply increases as more people can reach each
bike share station.

For this reason, we experiment with various walking thresholds by
conducting a sensitivity analysis to calculate accessibility at
different walking times: 3 minutes, 5 minutes, 10 minutes, and 15
minutes. We categorize these thresholds as minimum, average, maximum,
and extreme, respectively. At each threshold, we compare accessibility
between the current system and the original system to examine the
contribution of the additional equity stations.

\hypertarget{three-minutes}{%
\subsubsection{Three Minutes}\label{three-minutes}}

At this threshold, we find that there are 25.2 bicycle racks per person
in the original system. With the addition of equity stations, there are
now 25.4 bicycles per person. Figure \ref{fig:figure-6} presents a
comparison of accessibility between the systems. Accessibility is fairly
uniform, with the exception of two small areas near the university and
the waterfront park where accessibility is slightly higher. Equity
stations are shown in green, which helps to demonstrate how they
increase accessibility between the systems.

\begin{figure}

{\centering \includegraphics[width=0.65\linewidth]{Bike-share-spatial-equity_files/figure-latex/figure-6-1} 

}

\caption{Accessibility at 3 minutes walk (minimum threshold) compared between current system with equity stations and the original system without equity stations.}\label{fig:figure-6}
\end{figure}

\hypertarget{five-minutes}{%
\subsubsection{Five Minutes}\label{five-minutes}}

At this threshold, we find that there are 68.6 bicycle racks per person
in the original system. With the addition of equity stations, there are
now 68.8 bicycles per person. Figure \ref{fig:figure-7} presents a
comparison of accessibility between the systems. Again, accessibility is
fairly uniform, with the exception of one area near the waterfront park
where accessibility is higher.

\begin{figure}

{\centering \includegraphics[width=0.65\linewidth]{Bike-share-spatial-equity_files/figure-latex/figure-7-1} 

}

\caption{Accessibility at 5 minutes walk (average threshold) compared between current system with equity stations and the original system without equity stations.}\label{fig:figure-7}
\end{figure}

\hypertarget{ten-minutes}{%
\subsubsection{Ten Minutes}\label{ten-minutes}}

At this threshold, we find that there are 3.61 bicycle racks per person
in the original system. With the addition of equity stations, there are
now 3.74 bicycles per person. Figure \ref{fig:figure-8} presents a
comparison of accessibility between the systems. We begin to see
differences in accessibility across the service area, with users near
the university and its adjacent neighbourhoods, as well as
neighbourhoods north of the downtown area have slightly higher
accessibility. While the differences are modest, they are more apparent
at this threshold than at shorter walking distances.

\begin{figure}

{\centering \includegraphics[width=0.65\linewidth]{Bike-share-spatial-equity_files/figure-latex/figure-8-1} 

}

\caption{Accessibility at 10 minutes walk (maximum threshold) compared between current system with equity stations and the original system without equity stations.}\label{fig:figure-8}
\end{figure}

\hypertarget{fifteen-minutes}{%
\subsubsection{Fifteen Minutes}\label{fifteen-minutes}}

At this threshold, we find that there are 2.44 bicycle racks per person
in the original system. With the addition of equity stations, there are
now 2.55 bicycles per person. Figure \ref{fig:figure-9} presents a
comparison of accessibility between the systems. Users near the
university and the neighbourhoods north of the downtown area (outlined
in green) have the highest accessibility, followed by those who live in
the city's downtown area. Accessibility in the east end of the core
service area remains low.

\begin{figure}

{\centering \includegraphics[width=0.65\linewidth]{Bike-share-spatial-equity_files/figure-latex/figure-9-1} 

}

\caption{Accessibility at 15 minutes walk (extreme threshold) compared between current system with equity stations and the original system without equity stations.}\label{fig:figure-9}
\end{figure}

\hypertarget{accessibility-by-median-total-income}{%
\subsection{Accessibility by Median Total
Income}\label{accessibility-by-median-total-income}}

Approximately 138,000 people live within Hamilton Bike Share's core
service area or within a 10 minute walk to the service area. We find
that over 118,000 people can access a bicycle share station within a 15
minute walk, which represents roughly 85\% of the total population in
Hamilton Bike Share's core service area {[}see Table
\ref{tab:table-2}{]}. While previous research found that neighbourhoods
with more disadvantage are better serviced by Hamilton Bike Share
({\textbf{???}}), the authors used the Pampalon Deprivation Index from
2011 as a measure of socioeconomic status. Instead, we use median total
income for each dissemination area to examine whether disparities in
accessibility exist within the core service area. We divide income by
quantiles: bottom 20\%, second 20\%, third 20\%, fourth 20\%, and top
20\%. One of the unique properties of the balanced floating catchment
area method is that data can be reaggregated from small to larger
polygons while preserving the total population and number of bicycle
racks at each station. This avoids demand and supply inflation, and also
enables us to present findings in a way that is easier to interpret
(i.e., at the dissemination area level).

Panels 1 and 2 depict bivariate choropleth maps with the combined
spatial distribution of accessibility and median total income per
household for the different thresholds. Our analysis demonstrates that
the stations added to Hamilton's public bicycle share program to
increase equity for disadvantaged neighbourhoods achieved their goal by
increasing accessibility, albeit only modestly. By implementing equity
stations in more areas with lower median total income in Hamilton, the
PBSP has achieved greater horizontal equity by extending the spatial
distribution of bicycles across the city. This is particularly evident
at shorter walking distances of three and five minutes.

However, we found that dissemination areas with lower total household
income do not necessarily have greater access to the program {[}see
Table \ref{tab:table-3}{]}. Income disparities still persist, however
only at certain thresholds. With and without equity stations, people in
the top 20\% of income have the highest level of accessibility at a
threshold of ten and fifteen minutes. At distance thresholds of three
and five minutes, people in the second 20\% have the highest level of
accessibility by a significant amount. This is due to the concentration
of stations near or within the city's downtown area and the ``lower
city'' which historically has been more disadvantaged compared to the
surrounding suburban neighbourhoods. This is an artifact of the city's
steel industry which led to more blue collar and working class people
living in the city's downtown area and ``lower city''. However, the
bottom 20\%, who may benefit the most from Hamilton Bike Share's equity
initiatives, have the lowest accessibility at three minutes threshold
and the second lowest accessibility at all other thresholds.

On the whole, we find that areas that have less than their equitable
share of the level of service are some of the most socioeconomically
disadvantaged areas in the city. This finding aligns with other studies
that have found disparities in station or bike location along income
lines in New York City ({\textbf{???}}), Tampa ({\textbf{???}}), and
Seattle ({\textbf{???}}).

\begin{figure}
\includegraphics[width=1\linewidth]{Bike-share-spatial-equity_files/figure-latex/figure-bi-map-threshold-3-1} \caption{\label{fig-bivariate-map-threshold-3}Bivariate map of accessibility and income (threshold: 3 min): without equity stations (top panel) and with equity stations (bottom panel)}\label{fig:figure-bi-map-threshold-3}
\end{figure}

\begin{figure}
\includegraphics[width=1\linewidth]{Bike-share-spatial-equity_files/figure-latex/figure-bi-map-threshold-5-1} \caption{\label{fig-bivariate-map-threshold-5}Bivariate map of accessibility and income (threshold: 5 min): without equity stations (top panel) and with equity stations (bottom panel)}\label{fig:figure-bi-map-threshold-5}
\end{figure}

\begin{figure}
\includegraphics[width=1\linewidth]{Bike-share-spatial-equity_files/figure-latex/figure-bi-map-threshold-10-1} \caption{\label{fig-bivariate-map-threshold-10}Bivariate map of accessibility and income (threshold: 10 min): without equity stations (top panel) and with equity stations (bottom panel)}\label{fig:figure-bi-map-threshold-10}
\end{figure}

\begin{figure}
\includegraphics[width=1\linewidth]{Bike-share-spatial-equity_files/figure-latex/figure-bi-map-threshold-15-1} \caption{\label{fig-bivariate-map-threshold-15}Bivariate map of accessibility and income (threshold: 15 min): without equity stations (top panel) and equity stations (bottom panel)}\label{fig:figure-bi-map-threshold-15}
\end{figure}

\begin{table}

\caption{\label{tab:accessibility-income}\label{tab:accessibility-by-income}Accessibility and population serviced by income quintile}
\centering
\resizebox{\linewidth}{!}{
\begin{tabular}[t]{lcccccc}
\toprule
\multicolumn{1}{c}{ } & \multicolumn{2}{c}{Without Equity Stations} & \multicolumn{2}{c}{With Equity Stations} & \multicolumn{2}{c}{Difference} \\
\cmidrule(l{3pt}r{3pt}){2-3} \cmidrule(l{3pt}r{3pt}){4-5} \cmidrule(l{3pt}r{3pt}){6-7}
Income Quintile & Accessibility & Population & Accessibility & Population & Accessibility & Population\\
\midrule
\addlinespace[0.3em]
\multicolumn{7}{l}{\textbf{Threshold - 3 minutes}}\\
\hspace{1em}\cellcolor{gray!6}{Bottom 20\%} & \cellcolor{gray!6}{2.377} & \cellcolor{gray!6}{22359} & \cellcolor{gray!6}{2.424} & \cellcolor{gray!6}{22798} & \cellcolor{gray!6}{0.047} & \cellcolor{gray!6}{439}\\
\hspace{1em}Second 20\% & 12.203 & 9347 & 12.281 & 12420 & 0.078 & 3073\\
\hspace{1em}\cellcolor{gray!6}{Third 20\%} & \cellcolor{gray!6}{3.093} & \cellcolor{gray!6}{7745} & \cellcolor{gray!6}{3.156} & \cellcolor{gray!6}{9455} & \cellcolor{gray!6}{0.063} & \cellcolor{gray!6}{1710}\\
\hspace{1em}Fourth 20\% & 4.119 & 1673 & 4.119 & 1673 & 0.000 & 0\\
\hspace{1em}\cellcolor{gray!6}{Top 20\%} & \cellcolor{gray!6}{3.757} & \cellcolor{gray!6}{2151} & \cellcolor{gray!6}{3.784} & \cellcolor{gray!6}{2416} & \cellcolor{gray!6}{0.027} & \cellcolor{gray!6}{265}\\
\addlinespace[0.3em]
\multicolumn{7}{l}{\textbf{Threshold - 5 minutes}}\\
\hspace{1em}Bottom 20\% & 1.302 & 35477 & 1.357 & 35803 & 0.055 & 326\\
\hspace{1em}\cellcolor{gray!6}{Second 20\%} & \cellcolor{gray!6}{56.048} & \cellcolor{gray!6}{17513} & \cellcolor{gray!6}{56.137} & \cellcolor{gray!6}{22908} & \cellcolor{gray!6}{0.089} & \cellcolor{gray!6}{5395}\\
\hspace{1em}Third 20\% & 4.259 & 15117 & 4.291 & 18309 & 0.032 & 3192\\
\hspace{1em}\cellcolor{gray!6}{Fourth 20\%} & \cellcolor{gray!6}{1.094} & \cellcolor{gray!6}{2867} & \cellcolor{gray!6}{1.095} & \cellcolor{gray!6}{3116} & \cellcolor{gray!6}{0.001} & \cellcolor{gray!6}{249}\\
\hspace{1em}Top 20\% & 6.256 & 4074 & 6.264 & 4540 & 0.008 & 466\\
\addlinespace[0.3em]
\multicolumn{7}{l}{\textbf{Threshold - 10 minutes}}\\
\hspace{1em}\cellcolor{gray!6}{Bottom 20\%} & \cellcolor{gray!6}{0.604} & \cellcolor{gray!6}{41824} & \cellcolor{gray!6}{0.622} & \cellcolor{gray!6}{41981} & \cellcolor{gray!6}{0.018} & \cellcolor{gray!6}{157}\\
\hspace{1em}Second 20\% & 0.862 & 27546 & 0.929 & 30503 & 0.067 & 2957\\
\hspace{1em}\cellcolor{gray!6}{Third 20\%} & \cellcolor{gray!6}{0.776} & \cellcolor{gray!6}{22394} & \cellcolor{gray!6}{0.802} & \cellcolor{gray!6}{25128} & \cellcolor{gray!6}{0.026} & \cellcolor{gray!6}{2734}\\
\hspace{1em}Fourth 20\% & 0.225 & 4544 & 0.227 & 4989 & 0.002 & 445\\
\hspace{1em}\cellcolor{gray!6}{Top 20\%} & \cellcolor{gray!6}{1.346} & \cellcolor{gray!6}{7989} & \cellcolor{gray!6}{1.348} & \cellcolor{gray!6}{9078} & \cellcolor{gray!6}{0.002} & \cellcolor{gray!6}{1089}\\
\addlinespace[0.3em]
\multicolumn{7}{l}{\textbf{Threshold - 15 minutes}}\\
\hspace{1em}Bottom 20\% & 0.536 & 42208 & 0.557 & 42327 & 0.021 & 119\\
\hspace{1em}\cellcolor{gray!6}{Second 20\%} & \cellcolor{gray!6}{0.555} & \cellcolor{gray!6}{30507} & \cellcolor{gray!6}{0.614} & \cellcolor{gray!6}{31069} & \cellcolor{gray!6}{0.059} & \cellcolor{gray!6}{562}\\
\hspace{1em}Third 20\% & 0.554 & 26108 & 0.581 & 26660 & 0.027 & 552\\
\hspace{1em}\cellcolor{gray!6}{Fourth 20\%} & \cellcolor{gray!6}{0.093} & \cellcolor{gray!6}{6312} & \cellcolor{gray!6}{0.096} & \cellcolor{gray!6}{7435} & \cellcolor{gray!6}{0.003} & \cellcolor{gray!6}{1123}\\
\hspace{1em}Top 20\% & 0.808 & 10209 & 0.811 & 11089 & 0.003 & 880\\
\bottomrule
\multicolumn{7}{l}{\rule{0pt}{1em}\textit{Note: }}\\
\multicolumn{7}{l}{\rule{0pt}{1em} }\\
\multicolumn{7}{l}{\rule{0pt}{1em}\textsuperscript{a} With equity stations = Hamilton Bike Share current system (118 conventional stations, 12 equity stations)}\\
\multicolumn{7}{l}{\rule{0pt}{1em}\textsuperscript{b} Without equity stations = Hamilton Bike Share original system (118 conventional stations, no equity stations)}\\
\end{tabular}}
\end{table}

\hypertarget{study-limitations}{%
\section{Study Limitations}\label{study-limitations}}

This paper did not examine or compare ridership data between
conventional and equity stations. Therefore, further research is needed
to determine whether this encouraged more cycling for low-income
individuals living near equity stations. Other studies have specifically
looked at differences in trip type, frequency, or length among users
from disadvantaged neighbourhoods ({\textbf{???}}; {\textbf{???}};
{\textbf{???}}), but our analysis is limited by the lack of accessible
data to conduct similar analyses for Hamilton Bike Share.

\hypertarget{conclusion}{%
\section{Conclusion}\label{conclusion}}

The addition of specific equity stations to the public bicycle share
program in Hamilton, Ontario had the net effect of increasing
accessibility and reducing both horizontal and vertical inequities. In
particular, accessibility improved the most for the second 20\% income
groups at all thresholds, but the gains were only modest. Congestion
effects were observed at higher thresholds, with accessibility
decreasing significantly once the catchment area is increased to 10
minutes walking. However, We identified specific areas that have both
low accessibility and low income which would benefit from an increased
supply of bicycles. These are ideal candidates for new equity stations.
In this way, our paper has made contributions in a positive way by
applying an intuitive and useful approach to measure accessibility to a
PBSP, and in a normative way by serving to inform future investments in
cycle infrastructure for Hamilton Bike Share.

Wang and Lindsey ({\textbf{???}}) have noted that there is a lack of
research that examines how bike share users' behaviour changes as a
result of program changes to station locations or improvements in
accessibility. As such, a logical next step to this research is to
examine whether Hamilton Bike Share's equity stations increased
ridership or resulted in new memberships in areas that were previously
under-served. An examination of the types of trips undertaken by
residents in these areas would also be informative, such as the study
undertaken by Caspi and Norland ({\textbf{???}}) after bike share
stations were implemented in low-income Philadelphia neighbourhoods. The
bulk of cycling facilities that have been built in Hamilton to date
located in the core service area near the conventional stations. It
would be worthwhile to explore the route choice of bike share trips
departing or ending at the equity stations and to identify factors that
specifically influence trips from these stations, which would extend
existing studies conducted by Scott and colleagues ({\textbf{???}};
{\textbf{???}}; {\textbf{???}}). This paper, combined with additional
studies such as those conceptualized above, would serve as a valuable
case study for Hamilton and other cities with PBSPs who wish to evaluate
and address spatial inequities in accessibility and transportation
options in urban areas.

\hypertarget{references}{%
\section*{References}\label{references}}
\addcontentsline{toc}{section}{References}


\end{document}


