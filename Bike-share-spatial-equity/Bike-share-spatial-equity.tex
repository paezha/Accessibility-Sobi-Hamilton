\documentclass[]{elsarticle} %review=doublespace preprint=single 5p=2 column
%%% Begin My package additions %%%%%%%%%%%%%%%%%%%
\usepackage[hyphens]{url}

  \journal{Transportation Research Part D: Transport and Environment} % Sets Journal name


\usepackage{lineno} % add
\providecommand{\tightlist}{%
  \setlength{\itemsep}{0pt}\setlength{\parskip}{0pt}}

\usepackage{graphicx}
\usepackage{booktabs} % book-quality tables
%%%%%%%%%%%%%%%% end my additions to header

\usepackage[T1]{fontenc}
\usepackage{lmodern}
\usepackage{amssymb,amsmath}
\usepackage{ifxetex,ifluatex}
\usepackage{fixltx2e} % provides \textsubscript
% use upquote if available, for straight quotes in verbatim environments
\IfFileExists{upquote.sty}{\usepackage{upquote}}{}
\ifnum 0\ifxetex 1\fi\ifluatex 1\fi=0 % if pdftex
  \usepackage[utf8]{inputenc}
\else % if luatex or xelatex
  \usepackage{fontspec}
  \ifxetex
    \usepackage{xltxtra,xunicode}
  \fi
  \defaultfontfeatures{Mapping=tex-text,Scale=MatchLowercase}
  \newcommand{\euro}{€}
\fi
% use microtype if available
\IfFileExists{microtype.sty}{\usepackage{microtype}}{}
\bibliographystyle{elsarticle-harv}
\ifxetex
  \usepackage[setpagesize=false, % page size defined by xetex
              unicode=false, % unicode breaks when used with xetex
              xetex]{hyperref}
\else
  \usepackage[unicode=true]{hyperref}
\fi
\hypersetup{breaklinks=true,
            bookmarks=true,
            pdfauthor={},
            pdftitle={Examining spatial equity in accessibility to a public bicycle share program using a balanced floating catchment area approach},
            colorlinks=false,
            urlcolor=blue,
            linkcolor=magenta,
            pdfborder={0 0 0}}
\urlstyle{same}  % don't use monospace font for urls

\setcounter{secnumdepth}{0}
% Pandoc toggle for numbering sections (defaults to be off)
\setcounter{secnumdepth}{0}

% Pandoc citation processing

% Pandoc header
\usepackage{booktabs}
\usepackage{longtable}
\usepackage{array}
\usepackage{multirow}
\usepackage{wrapfig}
\usepackage{float}
\usepackage{colortbl}
\usepackage{pdflscape}
\usepackage{tabu}
\usepackage{threeparttable}
\usepackage{threeparttablex}
\usepackage[normalem]{ulem}
\usepackage{makecell}
\usepackage{xcolor}



\begin{document}
\begin{frontmatter}

  \title{Examining spatial equity in accessibility to a public bicycle share
program using a balanced floating catchment area approach}
    \author[Some School]{Elise Desjardins\corref{1}}
   \ead{desjae@mcmaster.ca} 
    \author[Another University]{Christopher D. Higgins\corref{2}}
   \ead{cd.higgins@utoronto.ca} 
    \author[Some School]{Antonio Paez\corref{2}}
   \ead{paezha@mcmaster.ca} 
      \address[McMaster University]{School of Earth, Environment \& Society, 1280 Main Street West,
Hamilton, ON L8S4L8}
    \address[University of Toronto Scarborough]{Department of Geography \& Planning, 1265 Military Trail, Toronto, ON
M1C1A4}
      \cortext[1]{Corresponding Author}
    \cortext[2]{Equal contribution}
  
  \begin{abstract}
  Public bicycle share programs (PBSPs) are implemented to promote cycling
  as a convenient and sustainable mode of transportation. These systems
  can also play a role in addressing transportation needs and advancing
  transportation equity if they make cycling more accessible to lower
  income or disadvantaged populations. In Ontario, the City of Hamilton
  launched a public bicycle share program in 2015 that currently has over
  900 operational bicycles and 130 docking stations. This system was
  expanded in 2018 by an equity initiative that added twelve docking
  stations with the explicit objective of increasing spatial equity in
  access. Since the cost of reaching a bicycle share station decreases the
  potential of accessing the program, the location of stations matters,
  and previous research found that Hamilton's public bicycle share program
  targeted well disadvantaged areas in the city. In this research, we
  revisit the case of Hamilton and investigate differentials in
  accessibility to bicycle share stations using a balanced floating
  catchment area (BFCA) accessibility approach. We compare accessibility
  with and without the equity stations to assess the effect of the
  initiative. We implement our analysis parting from micro zones to better
  reflect walking to a docking station, and conduct a sensitivity analysis
  at several walking time thresholds. We then reaggregate the estimated
  accessibility for further analysis using census data. The addition of
  equity stations increased the serviced population at every threshold
  examined, and although accessibility increased for the whole population,
  this increase was relatively modest especially for population in the
  bottom 20\% of median household income.
  \end{abstract}
  
 \end{frontmatter}

\newpage

\hypertarget{introduction}{%
\section{Introduction}\label{introduction}}

The potential of public bicycle share programs (PBSPs) to increase
cycling levels is but one of many reasons for implementing such programs
in urban areas (Hosford et al., 2018, 2019). As a healthy, inexpensive,
and convenient form of public transportation, shared bicycles can
encourage individuals to take up cycling for short local trips or first
and last mile trips to other public transportation instead of using
personal vehicles. These systems can also play a role in addressing
transportation needs and advancing transportation equity if they make
cycling more accessible to lower income or disadvantaged populations. As
these systems become increasingly common, their introduction has been
accompanied by a flurry of research that investigates them from the
perspective of spatial and transportation equity (e.g., Smith et al.,
2015; Hosford and Winters, 2018; Hull Grasso et al., 2020; Mooney et
al., 2019; Qian and Jaller, 2020; Qian et al., 2020). Indeed, although
these programs are available to the general public and ought to be
accessible to any individuals who wish to use them, research on PBSPs
located in North American and European cities indicates that there are
inequities in spatial accessibility in need of redress.

In this paper we investigate the case of the PBSP in the City of
Hamilton, in Ontario, Canada. Hamilton Bike Share (previously also known
as Social Bicycles, or SoBi) was launched in 2015 and currently has over
900 operational bicycles and 130 docking stations. An interesting
feature of this system is that it was expanded by an equity initiative
that introduced twelve docking stations in 2018 with the explicit
objective of increasing spatial accessibility. Hamilton Bike Share was
already studied by Hosford and Winters (2018) as part of a selection of
PBSPs in Canada. These authors found that most of the cities in their
research could benefit of greater efforts to expand service to areas
with lower socio-economic populations - but Hamilton fared somewhat
better than other cities in this respect. What distinguishes our
research is the use of a balanced floating catchment area (BFCA)
accessibility approach that accounts for the supply of stations and the
potential demand from the population serviced. This requires a more
disaggregated approach than the use of census geographies (q.v., Hosford
and Winters, 2018), since many trips to a bike share station likely
happen at a level lower than even the smallest census geography. For
this reason, we implement our analysis parting from micro population
zones to better reflect the friction of walking to a docking station.
Further, we conduct a sensitivity analysis at several walking time
thresholds. In terms of spatial equity, we compare accessibility with
and without the equity stations to assess the effect of the initiative,
before reaggregating the data for further analysis using median
household income information from the census. We find that the addition
of equity stations does increase the serviced population at every
threshold examined. However, although accessibility increased for the
whole population, this increase was relatively modest, especially for
population in the bottom 20\% of median household income. The analysis
also identifies areas under-serviced within the core service area of the
system.

This paper is an example of open and reproducible research that uses
only open software for transportation and statistical analysis (Bivand,
2020; Lovelace, 2021). All data were obtained from publicly available
sources and organized in the form a data package. Following best
practices in spatial data science (Brunsdon and Comber, 2020), the code
and data needed to reproduce (or modify/extend) the analysis are
available for download.\footnote{https://drive.google.com/drive/folders/1ZGRpSN2LxS2Fq2tLVJgfm1sI6I26vzZh?usp=sharing}

\hypertarget{literature-review}{%
\section{Literature Review}\label{literature-review}}

Public bicycle share programs have been implemented in over 800 cities
worldwide and a great deal has been learned about their typical users
(Fishman, 2016). In many cities, males use bike share more than females
({\textbf{???}}; Brey et al., 2017; Nickkar et al., 2019; Reilly, S. M.
Wang, et al., 2020; Winters et al., 2019) as do younger age cohorts
(Brey et al., 2017; Buck et al., 2013; Fuller et al., 2011). However,
one study found that bike share users in Washington, DC were more likely
to be female (Buck et al., 2013), which suggests that the gender gap
among cyclists who use bike share is less disparate than the gap for
personal bicycle use (Fishman, 2016). There is some evidence that bike
share users are less likely to own a car (Buck et al., 2013; Reilly,
Noyes, et al., 2020). However, the relationship between income or
education and bike share use is less clear-cut. Stations in
disadvantaged communities in Chicago have been found to generate most of
the average annual trips (Qian and Jaller, 2020) and individuals from
minority or lower socioeconomic status neighbourhoods in
Minneapolis-St.~Paul used the city's PBSP more (J. Wang and Lindsey,
2019a). Similar findings were reported in London ({\textbf{???}}). Being
university educated was a significant correlate of bike share use in
Montreal, Canada (Fuller et al., 2011). Perhaps not coincidentally,
financial savings have been found to motivate those on a low income to
use bike share (Fishman, 2016).

Many studies have found that geographic proximity to a bicycle share
station is an important determinant of membership and use (Fishman,
2016; Fuller et al., 2011). This makes sense given that individuals are
more likely to use services or programs that they can easily reach.
Several studies have recently explored the equity of PBSPs by primarily
examining who has access to bike share (e.g., differences by
demographics or socioeconomic status) and where stations are located. It
is important to note that equity can be achieved in two different ways:
equal spatial distribution across a region (e.g., horizontal equity) or
greater access for vulnerable or disadvantaged populations (e.g.,
vertical equity) (Chen et al., 2019). Both are of interest to
researchers and transport planners since they are often linked in that
advantage, or conversely disadvantage, has spatial patterns. Using a
negative binomial regression model, Qian and Jaller (2020) estimated
ridership in Chicago's PBSP among disadvantaged communities and found
some disparities. While annual members in disadvantaged communities have
a significantly lower share of trips compared to other areas in the
city, they make longer trips. This suggests that they may be using PBSPs
for utilitarian trips (e.g., commuting), which points to the importance
of ensuring equitable access (Qian and Jaller, 2020). Similar results
were found in Philadelphia, where lower income areas generated fewer
trips and it was suggested that efforts to increase equity within the
program have not been as successful as intended (Caspi and Noland,
2019). In the case of Seattle, all neighbourhoods had some level of
access to dockless bikes but those with higher incomes and more
residents of higher education had more bikes (Mooney et al., 2019).
Babagoli et al.~(2019) also found that neighbourhoods in New York City
with higher affluence had the greatest proportion of Citi Bike stations.
Researchers have recommended targeted expansion of stations to areas
that use bike share but have inequitable access ({\textbf{???}}).

On the whole, existing studies highlight the need for bicycle share
systems to be more highly accessible for diverse populations in order to
increase use beyond the ``typical'' users. This has been the focus of
recent research (see, among others, Auchincloss et al., 2020; Hull
Grasso et al., 2020; MacArthur et al., 2020). Offering more people the
option of using sustainable and active transportation, particularly
those who have lower socioeconomic status and might benefit the most, is
a worthy policy goal for cities with PBSPs. Several cities have launched
specific programs to address barriers to bike share or to expand service
to more deprived areas: London, United Kingdom ({\textbf{???}});
Philadelphia, Pennsylvania (Caspi and Noland, 2019), and Hamilton,
Ontario (Hosford and Winters, 2018). Findings to date suggest that
horizontal equity can be easier to achieve than vertical equity.
However, exploring transportation equity by investigating where bicycle
share stations are located, often using neighbourhoods or census tracts
as the geographical unit of analysis, can ignore or miss the benefits
that may be derived from adjacent zones. Meaning that, stations may be
lacking in certain neighbourhoods but there may be stations accessible
within a reasonable walking time. This is where geographical
accessibility becomes an important consideration.

Accessibility has been applied in both a positive and normative way to
inform transportation planning (Páez et al., 2012), but its utility to
this field has evolved over the past century and has increasingly become
linked with recent interest in prioritizing local proximity and modes
that are suitable for local trips like walking and cycling (Levine,
2020). As a measure of the ease of reaching potential destinations
spread spatially in a given area, accessibility is relevant to PBSPs
because it can identify current inequities in the provision of
infrastructure, as well as guide interventions that increase access for
groups that are under-serviced or address gaps in transportation
options. It also addresses some of the challenges of other performance
measures such as level of service within a transportation network by
measuring person-based indicators and exploring differences in use
between population groups (Páez et al., 2012).

Beyond the utility derived from using shared bikes to destinations of
value, an important aspect of accessibility associated with PBSPs is the
distance an individual must travel to reach a bicycle share station
(Kabra et al., 2020; J. Wang and Lindsey, 2019b). Since the time or
distance needed to reach a bicycle share station decreases the potential
of accessing the program and ultimately their use of the program, the
location and size (e.g., maximum number of bicycle racks available) of
stations matters. Indeed, distance to bicycle share stations is
associated with use (Fuller et al., 2011; J. Wang and Lindsey, 2019b)
and can be a barrier to using PBSPs (Fishman et al., 2014). Kabra et
al.~(2020) found that the majority of bike share usage in Paris comes
from areas within 300 metres of stations, which amounts to 2-4 minutes
walking by an able-bodied adult. Furthermore, similar to other public
amenities affected by crowding, like health care services (e.g.,
Pereira, Carlos Kauê Vieira Braga, et al., 2021), the utility of
stations is limited by the number of bicycles that they can hold. The
program may not necessarily be improved if stations are easy to reach
but offer only a small number of bicycles. Likewise, more people may not
opt to use the program if the supply of bicycles available at the
nearest station is insufficient to meet demand. The location and size of
stations is important to increase the utility of this public
transportation option for more people, thus achieving vertical equity.
Accessibility analyses for PBSPs constitute a positive and
evaluation-based approach that also has the potential to inform equity
efforts. For instance, Wang and Lindsey (2019b) investigated whether new
or relocated bicycle share stations increased accessibility and use,
which offered important insights to improve the performance of the
program.

Several approaches have been commonly used for measuring place-based
accessibility, including cumulative opportunities, gravity, and
utility-based measures (Handy and Niemeier, 1997). Geurs and Van Wee
(2004) and Paez et al.~(2012) provide recent overviews of various
formulations and applications of accessibility in transportation
research. The common gravity-based approach for example involves
weighting destination opportunities, such as the quantity of bicycle
share stations, by the time required to reach them from an origin using
an impedance function (Handy and Niemeier, 1997; Kwan, 1998). However,
while such measures are suitable for capturing the potential for
reaching destinations from a given location, they do not take demand or
congestion effects into account.

In contrast, floating catchment area (FCA) methods incorporate
information on capacity and demand in calculating accessibility. FCA
measures have been widely employed in healthcare accessibility research
and are typically calculated across two steps. In the first, a ratio of
supply to demand at service locations is calculated, such as the number
of beds at a hospital divided by the number of people within the
catchment area of the hospital, weighted by the distance involved in
reaching the facility. Next, these service level ratios are allocated
back to the population centres and summarized as a measure of congested
accessibility. Thus, this model does a good job of considering potential
crowding or competition for services. While there have been many
methodological innovations in FCA methods (for example, {\textbf{???}};
{\textbf{???}}; {\textbf{???}}; {\textbf{???}}; {\textbf{???}}), a
recent improvement to this approach was achieved through a simple and
intuitive balancing of the impedance that addressed the effects of
demand and service inflation found in previous FCA approaches (see Paez
et al., 2019).

When measuring accessibility, researchers have taken different
approaches when it comes to aggregation of data, either by using the
individual or household as the smallest unit of analysis or larger
spatial zones. Previous research on bike share equity has typically used
a meso- or macro-level approach with aggregate data from entire
neighbourhoods or census tracts (Babagoli et al., 2019; Mooney et al.,
2019; Qian and Jaller, 2020; J. Wang and Lindsey, 2019a), although there
are recent exceptions (Chen et al., 2019; Chen and Li, 2021). This is
also true for studies examining correlates of bike share demand (J. Wang
and Lindsey, 2019b). Handy and Niemeier (1997) note that using
disaggregated data in accessibility analyses provides a more accurate
estimate for individuals, which is useful for addressing vertical
inequities in PBSP usage.

Using the balanced floating catchment area method (Paez et al., 2019), a
novel approach that has not been used yet in cycling research, with
disaggregate population-level data, in this paper we examine
accessibility to a public bicycle share program in Hamilton, Ontario. We
(1) conduct a sensitivity analysis by measuring accessibility and level
of service to bike share stations at different walking time thresholds
to reach a station: 3 minutes, 5 minutes, 10 minutes, and 15 minutes;
(2) explore the contribution of specific stations that were added to
Hamilton's PBSP to reducing both horizontal and vertical inequities; and
(3) examine whether disparities in accessibility exist according to
median household income of dissemination areas within the core service
area.

\hypertarget{sec:study}{%
\section{Case Study}\label{sec:study}}

\hypertarget{original-system}{%
\subsection{Original System}\label{original-system}}

The focus of this study is the city of Hamilton, located in Ontario,
Canada. The city launched a public bicycle share program in March 2015
with 115 stations and 750 bicycles (Hamilton, 2015). Before June 2020,
the program was known as Social Bicycles or SoBi Hamilton but is now
called Hamilton Bike Share. Stations are spaced between 300 and 600
metres apart (Scott et al., 2021). The core service area spans 40 sq.km
of the city although it was planned to be 20 sq.km ({\textbf{???}}), and
roughly 138,000 people are within 30 minutes walking of a bike share
station {[}see Figure \ref{fig:hamilton-and-sobi-service-area}{]}. This
represents roughly one fifth of the total population of the Hamilton
Census Metropolitan Area according to the 2016 Canadian Census. The City
of Hamilton undertook a large public engagement campaign to validate the
locations of bicycle share stations that had been selected and to
crowdsource potential locations for additional locations
({\textbf{???}}). Most of the suggested locations were in the east end
of the core service area that lack stations or in neighbourhoods not
serviced by the PBSP. The program was enthusiastically welcomed in the
city in 2015 - within three weeks of launching, 10,000 trips had been
made (Hamilton, 2015), however inadequate coverage given the size of the
Hamilton Bike Share's service area was identified as a problem early on,
and transportation planners noted that the small size (i.e., supply of
bicycle racks) and low quantity of stations would lead to challenges in
balancing the system ({\textbf{???}}). Based on public feedback, 13
stations were added to the original system.

\hypertarget{equity-initiative}{%
\subsection{Equity Initiative}\label{equity-initiative}}

In 2017, Hamilton Bike Share Inc., the non-profit organization that
operates the program, initiated an equity program, Everyone Rides
Initiative (ERI), with the objective of reducing barriers that may
prevent individuals from accessing bike share in Hamilton. Additional
bicycles and stations were added to the program which expanded it to
more disadvantaged areas in the core service area {[}see Figure 2{]}.
The program also offers subsidized memberships to individuals who
identify as low income, and complements these initiatives with cycle
skills education. A comparable program can be found in Philadelphia (see
Caspi and Noland, 2019).

\hypertarget{current-system}{%
\subsection{Current System}\label{current-system}}

As of June 2020, Hamilton Bike Share has 900 bikes, 130 stations {[}see
Figure \ref{fig:sobi-stations-in-hamilton}{]}, and over 26,000 active
memberships (Hamilton, 2015). The core service area remains 40 sq.km.
The PBSP has 13 stations that were added as part of the ERI initiative;
we refer to these as ``equity stations'' throughout the paper while all
other stations are ``conventional stations''. In total, 117 stations are
``conventional'' and 13 are ``equity''.

\hypertarget{membership}{%
\subsection{Membership}\label{membership}}

Hamilton Bike Share conducted one membership survey in 2018 (Civicplan,
2017) and the findings from 420 members are broadly in line with the
trends that were discussed above (see Fishman, 2016 for a recent review
of the literature). The majority of respondents live within the core
service area and the gender split is expected: 57\% of respondents are
male and 41\% are female. The majority of respondents, both male and
female, are between 25 and 34 years of age, but the percentage of male
respondents is higher in the subsequent age groups. Respondents use bike
share for commuting (40\% of trips) or errands and meetings (24\% of
trips) and nearly 50\% of trips have an average length of 11 to 20
minutes. As a result of having a bike share membership, 49\% of
respondents report that they use their private vehicle less often or
much less often but 48\% report that their private vehicle use has
remained about the same. This suggests that Hamilton Bike Share has been
useful for certain kinds of trips but not all, meaning that some trips
continue to require a private vehicle.

\hypertarget{relevant-research}{%
\subsection{Relevant Research}\label{relevant-research}}

Our analysis builds upon a previous and recent study (Hosford and
Winters, 2018), which found that areas in Hamilton with less advantage
are better served by the city's PBSP compared to other Canadian cities
{[}i.e., Toronto, Vancouver, Montreal, and Ottawa-Gatineau{]} where
areas that are less deprived have greater access. Hosford and Winters
(2018) acknowledge that ``Hamilton stands out in that the lower income
neighborhoods are located near the city center and wealthier
neighborhoods are in the surrounding suburban areas''. Therefore, the
core service area for the PBSP in Hamilton by default covers some of the
less advantaged areas in the city. However, there is also a great deal
of variation in income in the city center because of the local
university and increasing gentrification. Hosford and Winters (2018)
took a macro-level and system-wide approach in their analysis by using
dissemination areas \emph{across} the city as the unit of analysis. They
did not focus specifically on the core service area and did not
differentiate between conventional and equity stations.

\begin{figure}

{\centering \includegraphics[width=0.9\linewidth]{Bike-share-spatial-equity_files/figure-latex/hamilton-and-sobi-service-area-1} 

}

\caption{The core service area of Hamilton Bike Share is outlined in black. Hamilton Census Metropolitan Area (CMA) is shown in grey.}\label{fig:hamilton-and-sobi-service-area}
\end{figure}

\begin{figure}

{\centering \includegraphics[width=1\linewidth]{Bike-share-spatial-equity_files/figure-latex/sobi-stations-in-hamilton-1} 

}

\caption{The spatial distribution of bike share stations in Hamilton, Ontario. The service area of Hamilton Bike Share is outlined in black and the city's downtown core is outlined in dark green. Hamilton CMA is shown in grey.}\label{fig:sobi-stations-in-hamilton}
\end{figure}

\hypertarget{sec:methods}{%
\section{Methods}\label{sec:methods}}

\hypertarget{floating-catchment-area}{%
\subsection{Floating Catchment Area}\label{floating-catchment-area}}

Floating catchment area (FCA) methods are an approach commonly used in
the healthcare accessibility literature. This approach is more
appropriate and informative than calculating provider-to-population
ratios (PPR) that simply divide the level of supply of a service (e.g.,
the number of bicycle racks at a station) by the population who have
access to the service (Paez et al., 2019). In particular, the Two-Step
Floating Catchment Area (2SFCA) method ({\textbf{???}}; {\textbf{???}})
produces flexible catchment areas instead of using rigid boundaries like
PPR. This provides more useful information because it does not assume
that people are limited to service within pre-defined boundaries (Paez
et al., 2019). This is an important property that supports our rationale
for applying this method to measure accessibility to Hamilton Bike
Share. The City of Hamilton has positioned stations between 300 and 600
metres apart, but anticipates that hubs will service those living within
a 250 metre buffer from the station ({\textbf{???}}). The latter
constitutes a normative statement: people ought to be able to access a
station in less than 600 metres if they live in the core service area,
with usage coming from 250 metres around. However, it is not known how
far people are actually willing to travel to reach a station. It would
be reasonable to assume that people are willing to walk beyond this
threshold to access other stations if the ones nearest them have no
supply of bicycles.

More recently, the \emph{balanced} floating catchment area (FCA)
approach was developed to address issues with demand and supply
inflation that result from the overlapping catchment areas produced by
earlier FCA methods (Paez et al., 2019). Briefly, overlapping catchment
areas lead to inflation of population totals and deflation of service
levels across a study area and generates inaccurate or misleading
accessibility estimates. In contrast, Paez et al.~(2019) adjust the
impedance weights so that both supply and demand are proportionally
allocated. The results is a FCA method that balances the population and
level of service by eliminating the over-counting of population and
level of service that leads to distortions in demand and supply. Other
benefits of this adjusted method include consideration of competition
which can occur when catchment areas overlap, as well as the
preservation of population and level of service. Balanced floating
catchment area methods have been used to explore accessibility to health
care providers (Paez et al., 2019) and COVID-19 health care services
(Pereira, Carlos Kauê Vieira Braga, et al., 2021), but have not yet been
used in the cycling literature to explore issues of accessibility.

The first step in the floating catchment area method is to allocate the
population to be serviced by each Hamilton Bike Share station: \[
P_j = {\sum_{i = 1}^{n} P_i{w_{ij}}}
\] As seen in the equation above, the population allocated to station
\(j\) is the weighted sum of the population in the region; a spatial
weight \(w_{ij}\) represents the friction that the population at \(i\)
faces when reaching station \(j\), and is usually given by a
distance-decay function, so that each station is assumed to service only
a segment of the population within a limited geographical range.

Next, the supply at each station (i.e., the maximum number of bicycle
racks) is divided by its estimated service population within the
established catchment area; this gives the level of service of station
\(j\) in bicycle racks per person: \[
L_j = \frac {S_j}{P_j} = \frac {S_j}{{\sum_{i = 1}^{n} P_i{w_{ij}}}}
\]

Finally, the accessibility of population cell \(i\) is calculated as the
weighted sum of the level of service of all stations that can be reached
from there according to the spatial weights: \[
A_i = {\sum_{j = 1}^{J} L_j{w_{ji}}} = {\sum_{j = 1}^{J} \frac {S_j{w_{ji}}}{\sum_{i = 1}^{n} P_i{w_{ij}}}}
\]

The balanced approach of Paez et al. (2019) replaces the spatial weights
with normalized versions as follows: \[
{w_{ij}^{i} = \frac {w_{ij}}{\sum_{j = 1}^{J} {w_{ji}}}}
\]

\noindent and: \[
{w_{ij}^{j} = \frac {w_{ij}}{\sum_{j = 1}^{J} {w_{ji}}}}
\]

These weights satisfy the following properties: \[
\sum_{j = 1}^{J} {w^i_{ji}} = 1
\]

\noindent and: \[
\sum_{i = 1}^{n} {w^j_{ji}} = 1
\]

With these weights accessibility can be calculated without risk of
demand or supply inflation: \[
A_i = {\sum_{j = 1}^{J} \frac {S_j{w^j_{ij}}}{\sum_{i = 1}^{n} P_i{w^i_{ij}}}}
\] By allocating the population and level of service proportionally,
this method preserves the values of the population and level of service
and provides a regional provider-to-population ratio since: \[
\sum_{i=1}^n A_i = \sum_{j=1}^J L_j = \frac{\sum_{j=1}^JS_j}{\sum_{i=1}^n P_i}
\]

In fact, since the proportional allocation procedure means that any
proportion of the population allocated to a station is never allocated
to other stations, and conversely any level of service allocated to a
population is never re-allocated elsewhere, this property is replicated
for any level of aggregation. For this paper, we employ a hybrid
location-based and person-based approach to calculating accessibility
using disaggregate population data. In their review of accessibility
measures, Geurs and van Wee (2004) highlight the need for greater
inclusion of individual spatio-temporal constraints but acknowledge the
challenges of acquiring and analyzing person-based data. This comes
after Kwan's (1998) work to show that space-time measures are better
able of capturing interpersonal differences, especially the effect of
space-time constraints on individual behaviour, and are more helpful for
unraveling gender/ethnic differences. Applying the balanced floating
catchment area approach allows us to examine accessibility by
stratifying according to median household income, which would constitute
the individual component of the accessibility measure (Geurs and van
Wee, 2004). However, conducting a further sensitivity analysis to
measure accessibility at different distance thresholds would help to
consider potential spatio-temporal constraints. Different people may be
willing to travel different distances to access a public bicycle share
station.

\hypertarget{pycnophylactic-interpolation}{%
\subsection{Pycnophylactic
Interpolation}\label{pycnophylactic-interpolation}}

To obtain population at sub-census geography levels (at the micro scale)
we use pycnophylactic interpolation (Tobler, 1979). We obtained
population data from the 2016 Census of Canada for dissemination areas
(DAs, the smallest publicly available census geography in Canada). This
zonal values of the population were interpolated to smaller polygons 50
by 50 metres in size. Pycnophilactic interpolation involves smoothing
out the population from each dissemination area while preserving total
volume {[}see Figure \ref{fig:da-population}{]}. When interpolating the
population at this high level of resolution it is important to ensure
that population numbers were not allocated to areas where people do not
live in Hamilton (e.g., to parks, large institutional buildings, etc.)
To do so, we retrieved shapefiles for various geographic features {[}see
Table \ref{tab:data-features}{]} from Open Hamilton. Next, we removed
these features from the PBSP core service area and used pycnophylactic
interpolation to disaggregate and reallocate population within the
remaining area {[}see Figure \ref{fig:interpolated-population}{]}.

\begin{table}

\caption{\label{tab:data-features}\label{tab:landscape-features}Landscape features extracted from the Hamilton Bike Share core service area before population is interpolated.}
\centering
\resizebox{\linewidth}{!}{
\begin{tabular}[t]{>{}l|>{\raggedright\arraybackslash}p{30em}}
\toprule
Feature & Description\\
\midrule
\cellcolor{gray!6}{\textbf{Hamilton Bike Share Stations}} & \cellcolor{gray!6}{The location of stations and the number of racks available at each station.}\\
\textbf{Golf Courses} & The location of City and privately owned golf courses.\\
\cellcolor{gray!6}{\textbf{Parks}} & \cellcolor{gray!6}{The location of parks and other green spaces.}\\
\textbf{Designated Large Employment Areas} & The location of large business parks and industrial lands.\\
\cellcolor{gray!6}{\textbf{Municipal Parking Lots}} & \cellcolor{gray!6}{The location of municipal car parks.}\\
\addlinespace
\textbf{Cemeteries} & The location of cemeteries.\\
\cellcolor{gray!6}{\textbf{Environmentally Sensitive Areas}} & \cellcolor{gray!6}{The location of either land or water areas containing natural features or significant ecological functions.}\\
\textbf{Streets} & The street network in Hamilton, including road classification for highways.\\
\cellcolor{gray!6}{\textbf{Educational Institutions}} & \cellcolor{gray!6}{The location of all educational institutions and schools.}\\
\textbf{Places of Worship} & The location of buildings used for religious congregations.\\
\addlinespace
\cellcolor{gray!6}{\textbf{Municipal Service Centres}} & \cellcolor{gray!6}{The location of all municipal service centres, including City Hall.}\\
\textbf{Recreation and Community Centres} & The location of all recreation and community centres.\\
\cellcolor{gray!6}{\textbf{Arenas}} & \cellcolor{gray!6}{The location of all indoor arenas.}\\
\textbf{Emergency Stations} & The location of all Emergency Management Services (EMS) Ambulance stations.\\
\cellcolor{gray!6}{\textbf{Fire Stations}} & \cellcolor{gray!6}{The location of all fire stations.}\\
\addlinespace
\textbf{Police Stations} & The location of all police stations.\\
\cellcolor{gray!6}{\textbf{Railways}} & \cellcolor{gray!6}{The railway network in Hamilton.}\\
\textbf{Hospitals} & The location of all hospitals.\\
\bottomrule
\end{tabular}}
\end{table}

\begin{figure}

{\centering \includegraphics[width=0.9\linewidth]{Bike-share-spatial-equity_files/figure-latex/da-population-1} 

}

\caption{Population in all dissemination areas (outlined in white) that are inside or touch the bounding box of the Hamilton Bike Share core service area (outlined in black).}\label{fig:da-population}
\end{figure}

\begin{figure}

{\centering \includegraphics[width=0.9\linewidth]{Bike-share-spatial-equity_files/figure-latex/interpolated-population-1} 

}

\caption{Interpolated population in the Hamilton Bike Share core service area (outlined in black) and within 30 minutes of walking to the core service area (outlined in blue). Each micro population cell is 50-by-50 m in size. The downtown area is outlined in dark green.}\label{fig:interpolated-population}
\end{figure}

\hypertarget{travel-time-matrix}{%
\subsection{Travel Time Matrix}\label{travel-time-matrix}}

To calculate walking times from the centroid of our micro population
cells we extracted OpenStreetMap data for Hamilton Bike Share's service
area from \href{https://download.bbbike.org/osm/bbbike/}{BBBike}, an
online cycle route planner that interfaces with OpenStreetMap.
OpenStreetMap data provides the networks for calculate walking times
from each population cell to nearby bike share stations, using a maximum
walking distance of 10 km and walking time of 30 minutes as thresholds.
A travel time matrix was created with the origins as the coordinates of
the population cells and the destinations as the coordinates of the bike
share stations within the maximum threshold. This process provides a
more realistic measure of the friction of reaching bike share stations
by taking infrastructure into account of travel times, rather than using
the Euclidean distance from population cell to station. Routing and
travel time calculations were completed using the \texttt{R} package
\texttt{r5r}, used for rapid realistic routing operations (Pereira,
Saraiva, et al., 2021).

\hypertarget{data}{%
\subsection{Data}\label{data}}

All data for this research were accessed from publicly available Census
of Canada sources, from OpenStreetMaps, and from Open Hamilton\footnote{https://open.hamilton.ca/},
a public online repository of data curated by the City of Hamilton.

\hypertarget{results}{%
\section{Results}\label{results}}

\hypertarget{accessibility-by-distance-thresholds}{%
\subsection{Accessibility by Distance
Thresholds}\label{accessibility-by-distance-thresholds}}

Consensus regarding the distance that individuals are willing to walk to
access a PBSP station is lacking, but the literature on walking
behaviour was consulted to determine the thresholds for the sensitivity
analysis. Previous studies have found that living within 250 metres
(Fuller et al., 2011) and 300 metres (Kabra et al., 2020) is correlated
with bike share use, while other research has found that walking trips
are less than 600 metres and rarely more than 1200 metres (Millward et
al., 2013) or a median distance of 650 metres (Larsen et al., 2010). In
Hamilton, Hamilton Bike Share will depict a map at some stations to show
the user the locations of the other nearest stations within a five
minute walk, which suggests that this is an average distance that people
are expected to walk.

In the present case, we found that congested accessibility calculated
using the balanced FCA approach increases with a threshold between two
and four minutes, but is then maximized at 5 minutes. Accessibility
decreases substantially after eight minutes, which is intuitive given
that demand on a limited supply increases as more people can reach each
bike share station.

For this reason, we experiment with various walking thresholds by
conducting a sensitivity analysis to calculate accessibility at
different walking times from population cell to bicycle share station: 3
minutes, 5 minutes, 10 minutes, and 15 minutes. We categorize these
thresholds as minimum, average, maximum, and extreme, respectively. At
each threshold, we compare accessibility between the current system and
the original system to examine the contribution of the additional equity
stations.

\hypertarget{minimum-threshold}{%
\subsubsection{Minimum Threshold}\label{minimum-threshold}}

With a walking distance of three minutes, we find that there are 25.2
bicycle racks per person in the original system. The addition of equity
stations increases this ratio slightly to 25.4 bicycles per person.
Figure \ref{fig:figure-6} presents a comparison of accessibility between
the systems for the population cells. Accessibility is fairly uniform
overall, with the exception of two small areas where accessibility is
slightly higher.

\begin{figure}
 
 {\centering \includegraphics[width=0.9\linewidth]{Bike-share-spatial-equity_files/figure-latex/figure-6-1} 
 
 }
 
 \caption{Accessibility at 3 minutes walk (minimum threshold) compared between current system with equity stations and the original system without equity stations.}\label{fig:figure-6}
 \end{figure}

\hypertarget{average-threshold}{%
\subsubsection{Average Threshold}\label{average-threshold}}

With a walking distance of five minutes, we find that there are 68.6
bicycle racks per person in the original system. With the addition of
equity stations, there are now 68.8 bicycles per person. Figure
\ref{fig:figure-7} presents a comparison of accessibility between the
systems. Again, accessibility is fairly uniform, with the exception of
one very small area. At this threshold, there are more bicycle racks per
person than at the minimum threshold. This occurs because fewer stations
can be reached when the travel time is 3 minutes or less.

\begin{figure}

{\centering \includegraphics[width=0.9\linewidth]{Bike-share-spatial-equity_files/figure-latex/figure-7-1} 

}

\caption{Accessibility at 5 minutes walk (average threshold) compared between current system with equity stations and the original system without equity stations.}\label{fig:figure-7}
\end{figure}

\hypertarget{maximum-threshold}{%
\subsubsection{Maximum Threshold}\label{maximum-threshold}}

With a walking distance of ten minutes, we find that there are 3.61
bicycle racks per person in the original system. With the addition of
equity stations, there are now 3.74 bicycles per person. Figure
\ref{fig:figure-8} presents a comparison of accessibility between the
systems. We begin to see differences in accessibility across the service
area, with users near the university and its adjacent neighbourhoods, as
well as neighbourhoods north of the downtown area (outlined in green),
having slightly higher accessibility. While the differences are modest,
they are more apparent at this threshold than at shorter walking
distances.

\begin{figure}

{\centering \includegraphics[width=0.9\linewidth]{Bike-share-spatial-equity_files/figure-latex/figure-8-1} 

}

\caption{Accessibility at 10 minutes walk (maximum threshold) compared between current system with equity stations and the original system without equity stations.}\label{fig:figure-8}
\end{figure}

\hypertarget{extreme-threshold}{%
\subsubsection{Extreme Threshold}\label{extreme-threshold}}

With a walking distance of fifteen minutes, we find that there are 2.44
bicycle racks per person in the original system. With the addition of
equity stations, there are now 2.55 bicycles per person. Figure
\ref{fig:figure-9} presents a comparison of accessibility between the
systems. Users near the university and the neighbourhoods north of the
downtown area (outlined in green) have the highest accessibility,
followed by those who live in the city's downtown area. Accessibility in
the east end of the core service area remains low.

\begin{figure}

{\centering \includegraphics[width=0.9\linewidth]{Bike-share-spatial-equity_files/figure-latex/figure-9-1} 

}

\caption{Accessibility at 15 minutes walk (extreme threshold) compared between current system with equity stations and the original system without equity stations.}\label{fig:figure-9}
\end{figure}

\hypertarget{accessibility-by-median-household-income}{%
\subsection{Accessibility by Median Household
Income}\label{accessibility-by-median-household-income}}

One of the unique properties of the balanced floating catchment area
method is that data can be reaggregated while preserving the total
population in the service area and the supply at each station. This
avoids demand and supply inflation, and also enables us to present
findings in a way that is easier and more intuitive to interpret.
Therefore, we reaggregate population and accessibility from the smallest
50 by 50 metres population cells to the median household income for each
dissemination area. We divide income by quintiles: bottom 20\%, second
20\%, third 20\%, fourth 20\%, and top 20\%.

Panels 1 {[}\ref{fig:figure-bi-map-threshold-3}{]}, 2
{[}\ref{fig:figure-bi-map-threshold-5}{]}, 3
{[}\ref{fig:figure-bi-map-threshold-10}{]}, and 4
{[}\ref{fig:figure-bi-map-threshold-15}{]} depict bivariate choropleth
maps with the combined spatial distribution of accessibility and median
household income at different thresholds. Our analysis demonstrates that
the extreme threshold of fifteen minutes allows the most people to
access a station. We find that stations added to Hamilton's public
bicycle share program to increase equity for disadvantaged
neighbourhoods achieved their goal by increasing access, albeit only
modestly {[}see Table \ref{tab:accessibility-income}{]}. By implementing
equity stations in more areas with lower median total income in
Hamilton, the PBSP has achieved greater horizontal equity by extending
the spatial distribution of bicycles across the city. This is
particularly evident at the minimum and average thresholds of three and
five minutes, respectively, where the equity stations fill a number of
gaps in program coverage. The largest gains were made for dissemination
areas in the second 20\%, where an additional 3,073 and 5,395 people
could reach a bicycle share station within three and five minutes walk,
respectively, after the addition of the equity station.

However, we found that dissemination areas with the lowest median
household income do not have much greater access to the program {[}see
Table \ref{tab:table-3}{]}. Income disparities still persist, however
only at certain thresholds. With and without equity stations, people in
the top 20\% of income have the highest level of access at a threshold
of ten and fifteen minutes. Although dissemination areas in the second
20\% have the highest level of access by a significant amount at lower
distance thresholds, the bottom 20\%, who may benefit the most from
Hamilton Bike Share's equity initiatives, have the lowest access at
three minutes threshold and the second lowest access at all other
thresholds.

\begin{figure}
\includegraphics[width=1\linewidth]{Bike-share-spatial-equity_files/figure-latex/figure-bi-map-threshold-3-1} \caption{\label{fig-bivariate-map-threshold-3}Bivariate map of accessibility and income (threshold: 3 min): without equity stations (top panel) and with equity stations (bottom panel)}\label{fig:figure-bi-map-threshold-3}
\end{figure}

\begin{figure}
\includegraphics[width=1\linewidth]{Bike-share-spatial-equity_files/figure-latex/figure-bi-map-threshold-5-1} \caption{\label{fig-bivariate-map-threshold-5}Bivariate map of accessibility and income (threshold: 5 min): without equity stations (top panel) and with equity stations (bottom panel)}\label{fig:figure-bi-map-threshold-5}
\end{figure}

\begin{figure}
\includegraphics[width=1\linewidth]{Bike-share-spatial-equity_files/figure-latex/figure-bi-map-threshold-10-1} \caption{\label{fig-bivariate-map-threshold-10}Bivariate map of accessibility and income (threshold: 10 min): without equity stations (top panel) and with equity stations (bottom panel)}\label{fig:figure-bi-map-threshold-10}
\end{figure}

\begin{figure}
\includegraphics[width=1\linewidth]{Bike-share-spatial-equity_files/figure-latex/figure-bi-map-threshold-15-1} \caption{\label{fig-bivariate-map-threshold-15}Bivariate map of accessibility and income (threshold: 15 min): without equity stations (top panel) and equity stations (bottom panel)}\label{fig:figure-bi-map-threshold-15}
\end{figure}

\begin{table}

\caption{\label{tab:accessibility-income}\label{tab:accessibility-by-income}Accessibility and population serviced by income quintile and between systems (with and without equity stations).}
\centering
\resizebox{\linewidth}{!}{
\begin{tabular}[t]{lcccccc}
\toprule
\multicolumn{1}{c}{ } & \multicolumn{2}{c}{Without Equity Stations} & \multicolumn{2}{c}{With Equity Stations} & \multicolumn{2}{c}{Difference} \\
\cmidrule(l{3pt}r{3pt}){2-3} \cmidrule(l{3pt}r{3pt}){4-5} \cmidrule(l{3pt}r{3pt}){6-7}
Income Quintile & Accessibility & Population & Accessibility & Population & Accessibility & Population\\
\midrule
\addlinespace[0.3em]
\multicolumn{7}{l}{\textbf{Threshold - 3 minutes}}\\
\hspace{1em}\cellcolor{gray!6}{Bottom 20\%} & \cellcolor{gray!6}{2.377} & \cellcolor{gray!6}{22359} & \cellcolor{gray!6}{2.424} & \cellcolor{gray!6}{22798} & \cellcolor{gray!6}{0.047} & \cellcolor{gray!6}{439}\\
\hspace{1em}Second 20\% & 12.203 & 9347 & 12.281 & 12420 & 0.078 & 3073\\
\hspace{1em}\cellcolor{gray!6}{Third 20\%} & \cellcolor{gray!6}{3.093} & \cellcolor{gray!6}{7745} & \cellcolor{gray!6}{3.156} & \cellcolor{gray!6}{9455} & \cellcolor{gray!6}{0.063} & \cellcolor{gray!6}{1710}\\
\hspace{1em}Fourth 20\% & 4.119 & 1673 & 4.119 & 1673 & 0.000 & 0\\
\hspace{1em}\cellcolor{gray!6}{Top 20\%} & \cellcolor{gray!6}{3.757} & \cellcolor{gray!6}{2151} & \cellcolor{gray!6}{3.784} & \cellcolor{gray!6}{2416} & \cellcolor{gray!6}{0.027} & \cellcolor{gray!6}{265}\\
\addlinespace[0.3em]
\multicolumn{7}{l}{\textbf{Threshold - 5 minutes}}\\
\hspace{1em}Bottom 20\% & 1.302 & 35477 & 1.357 & 35803 & 0.055 & 326\\
\hspace{1em}\cellcolor{gray!6}{Second 20\%} & \cellcolor{gray!6}{56.048} & \cellcolor{gray!6}{17513} & \cellcolor{gray!6}{56.137} & \cellcolor{gray!6}{22908} & \cellcolor{gray!6}{0.089} & \cellcolor{gray!6}{5395}\\
\hspace{1em}Third 20\% & 4.259 & 15117 & 4.291 & 18309 & 0.032 & 3192\\
\hspace{1em}\cellcolor{gray!6}{Fourth 20\%} & \cellcolor{gray!6}{1.094} & \cellcolor{gray!6}{2867} & \cellcolor{gray!6}{1.095} & \cellcolor{gray!6}{3116} & \cellcolor{gray!6}{0.001} & \cellcolor{gray!6}{249}\\
\hspace{1em}Top 20\% & 6.256 & 4074 & 6.264 & 4540 & 0.008 & 466\\
\addlinespace[0.3em]
\multicolumn{7}{l}{\textbf{Threshold - 10 minutes}}\\
\hspace{1em}\cellcolor{gray!6}{Bottom 20\%} & \cellcolor{gray!6}{0.604} & \cellcolor{gray!6}{41824} & \cellcolor{gray!6}{0.622} & \cellcolor{gray!6}{41981} & \cellcolor{gray!6}{0.018} & \cellcolor{gray!6}{157}\\
\hspace{1em}Second 20\% & 0.862 & 27546 & 0.929 & 30503 & 0.067 & 2957\\
\hspace{1em}\cellcolor{gray!6}{Third 20\%} & \cellcolor{gray!6}{0.776} & \cellcolor{gray!6}{22394} & \cellcolor{gray!6}{0.802} & \cellcolor{gray!6}{25128} & \cellcolor{gray!6}{0.026} & \cellcolor{gray!6}{2734}\\
\hspace{1em}Fourth 20\% & 0.225 & 4544 & 0.227 & 4989 & 0.002 & 445\\
\hspace{1em}\cellcolor{gray!6}{Top 20\%} & \cellcolor{gray!6}{1.346} & \cellcolor{gray!6}{7989} & \cellcolor{gray!6}{1.348} & \cellcolor{gray!6}{9078} & \cellcolor{gray!6}{0.002} & \cellcolor{gray!6}{1089}\\
\addlinespace[0.3em]
\multicolumn{7}{l}{\textbf{Threshold - 15 minutes}}\\
\hspace{1em}Bottom 20\% & 0.536 & 42208 & 0.557 & 42327 & 0.021 & 119\\
\hspace{1em}\cellcolor{gray!6}{Second 20\%} & \cellcolor{gray!6}{0.555} & \cellcolor{gray!6}{30507} & \cellcolor{gray!6}{0.614} & \cellcolor{gray!6}{31069} & \cellcolor{gray!6}{0.059} & \cellcolor{gray!6}{562}\\
\hspace{1em}Third 20\% & 0.554 & 26108 & 0.581 & 26660 & 0.027 & 552\\
\hspace{1em}\cellcolor{gray!6}{Fourth 20\%} & \cellcolor{gray!6}{0.093} & \cellcolor{gray!6}{6312} & \cellcolor{gray!6}{0.096} & \cellcolor{gray!6}{7435} & \cellcolor{gray!6}{0.003} & \cellcolor{gray!6}{1123}\\
\hspace{1em}Top 20\% & 0.808 & 10209 & 0.811 & 11089 & 0.003 & 880\\
\bottomrule
\multicolumn{7}{l}{\rule{0pt}{1em}\textit{Note: }}\\
\multicolumn{7}{l}{\rule{0pt}{1em} }\\
\multicolumn{7}{l}{\rule{0pt}{1em}\textsuperscript{a} With equity stations = Hamilton Bike Share current system (118 conventional stations, 12 equity stations)}\\
\multicolumn{7}{l}{\rule{0pt}{1em}\textsuperscript{b} Without equity stations = Hamilton Bike Share original system (118 conventional stations, no equity stations)}\\
\end{tabular}}
\end{table}

\hypertarget{discussion}{%
\section{Discussion}\label{discussion}}

Using disaggregate data, we examined spatial equity and accessibility to
Hamilton Bike Share, with a particular focus on assessing the
contribution of the program's equity stations. Our balanced floating
catchment area approach, combined with pycnophylactic interpolation,
enabled us to measure accessibility on a micro sale. This differentiates
our analysis from similar papers exploring equity in PBSPs that use
larger geographical units of analysis or that focus on station location
instead of level of service. In this way, our paper has made
contributions in a positive way by applying an intuitive and useful
approach to measure accessibility to a PBSP, and in a normative way by
serving to inform future investments in cycle infrastructure for
Hamilton Bike Share.

The sensitivity analysis revealed that accessibility to Hamilton Bike
Share stations is maximized at five minutes and decreases significantly
by eight minutes {[}see Table \ref{tab:accessibility-income}{]}. This
reflects the normative guide, advertised on some bicycle share stations
in Hamilton, that people can access other stations within a five minute
walk if the station of origin has no bicycles. We find that over 118,000
people can access a bicycle share station within a 15 minute walk, which
represents roughly 85\% of the total population in the core service area
{[}see Table \ref{tab:accessibility-income}{]}. At a minimum threshold
of three minutes, too few stations can be accessed which leads to lower
levels of accessibility. However, accessibility is at its lowest after
eight minutes whereby congestion occurs due to increased potential
demand. The City of Hamilton has recognized from the launch of the PBSP
that significantly more stations and bicycles are needed to service the
area ({\textbf{???}}). With a service area of 40 sq.km, Hamilton should
have between 380 and 440 stations, instead of 130, and 1,500 bicycles
instead of 900 ({\textbf{???}}). Reduced capacity within the system
leads to gaps in coverage in some areas of the city ``with some areas
not having the recommended station density of 300m between stations or
10 stations per square km'' ({\textbf{???}}). This study illustrates the
consequences of an imbalanced system whereby levels of accessibility are
not equitable across income groups and become very low when congestion
occurs.

We found that equity stations achieved their goal of increasing
horizontal equity in Hamilton's core service area. Panels
1{[}\ref{fig:figure-bi-map-threshold-3}{]}, 2
{[}\ref{fig:figure-bi-map-threshold-5}{]}, 3
{[}\ref{fig:figure-bi-map-threshold-10}{]}, and 4
{[}\ref{fig:figure-bi-map-threshold-15}{]} demonstrate how gaps in
service were filled by these additional stations. In this respect, there
are some commonalities between the expansions of Hamilton Bike Share and
that of Citi Bike in New York. Babgoli et al.~(2019) found a slight but
not statistically significant increase in the proportion of
neighbourhoods with the highest levels of poverty with stations after
the Citi Bike expansion in 2015. Although the Citi Bike expansion was
not specifically driven by a desire to reduce inequities in access, 16\%
of neighbourhoods with the most poverty had stations compared to 12\%
before. With the addition of equity stations, there were large gains in
accessibility for the second 20\% at the average threshold, over 5,000
more people, but much smaller gains for the bottom 20\% with only 326
more people.

Vertical inequities, however, continue to persist despite the addition
of equity stations. While the addition of equity stations modestly
increases accessibility for all income groups at all thresholds, they
did not increase accessibility significantly for any single income
group. Most importantly, individuals in the bottom 20\% median household
income quintile have the second lowest level of access to Hamilton Bike
Share at most thresholds (average, maximum, and extreme). At the minimum
threshold, the bottom 20\% have the lowest level of access. While
previous research found that neighbourhoods with more disadvantage are
better serviced by Hamilton Bike Share, the authors used the Pampalon
Deprivation Index to determine the level of disadvantage for
dissemination areas \emph{across} the city not just within the core
service area (Hosford and Winters, 2018). Instead, we use median
household income for each dissemination area \emph{within} the core
service area. We conclude that Hamilton's PBSP, while by default located
in areas with more deprivation compared to other cities, has disparities
in accessibility between income groups. With equity stations, many areas
with low median household income in the east end of the service area
continue to have low accessibility. At higher thresholds of 10 and 15
minutes, the top 20\% have the highest level of access to Hamilton Bike
Share. This aligns with other studies from Tampa (Chen et al., 2019) and
Seattle (Mooney et al., 2019), which have found disparities in station
location or access to bicycles between levels of income and education.

Based on our analysis, we identified specific areas that have both low
accessibility and low median household income which would benefit from
an increased supply of public bicycles. These empirical findings provide
support to the City of Hamilton's efforts to increase equity and balance
the system, and confirm that additional stations and bicycles are needed
to improve access not only for the bottom 20\% but for all income
groups. Panels 1{[}\ref{fig:figure-bi-map-threshold-3}{]}, 2
{[}\ref{fig:figure-bi-map-threshold-5}{]}, 3
{[}\ref{fig:figure-bi-map-threshold-10}{]}, and 4
{[}\ref{fig:figure-bi-map-threshold-15}{]} highlight potential locations
for new equity stations.

\hypertarget{study-limitations}{%
\section{Study Limitations}\label{study-limitations}}

This paper did not examine or compare ridership data between
conventional and equity stations. Therefore, further research is needed
to determine whether the addition of equity stations encouraged more
cycling for low-income individuals living near them. Other studies have
specifically looked at differences in trip type, frequency, or length
among users from disadvantaged neighbourhoods (Caspi and Noland, 2019;
Qian and Jaller, 2020; J. Wang and Lindsey, 2019a), but our analysis is
limited by the lack of publicly available route or user data to conduct
similar analyses for Hamilton Bike Share.

\hypertarget{conclusion}{%
\section{Conclusion}\label{conclusion}}

The addition of specific equity stations to the public bicycle share
program in Hamilton, Ontario had the net effect of increasing
accessibility and reducing both horizontal and vertical inequities. In
particular, accessibility improved the most for those in the second 20\%
median household income at all thresholds, but the gains were only
modest for all income groups. Dissemination areas with the bottom 20\%
had the lowest accessibility at three minutes, and second lowest levels
of accessibility at five, ten, and fifteen minutes. Congestion effects
were observed at higher thresholds, with accessibility decreasing
significantly once the catchment area is increased to 10 minutes
walking.

Wang and Lindsey (2019b) have noted that there is a lack of research
that examines how bike share users' behaviour changes as a result of
program changes to station locations or improvements in accessibility.
As such, a logical next step to this research is to examine whether
Hamilton Bike Share's equity stations increased ridership or resulted in
new memberships in areas that were previously under-served. An
examination of the types of trips undertaken by residents in these areas
would also be informative, such as the study undertaken by Caspi and
Norland (2019) after bike share stations were implemented in low-income
Philadelphia neighbourhoods. The bulk of cycling facilities that have
been built in Hamilton to date located in the core service area near the
conventional stations. It would be worthwhile to explore the route
choice of bike share trips departing or ending at the equity stations
and to identify factors that specifically influence trips from these
stations, which would extend existing studies conducted by Scott and
colleagues (Lu et al., 2018; Scott and Ciuro, 2019; Scott et al., 2021).
This paper, combined with additional studies such as those
conceptualized above, would serve as a valuable case study for Hamilton
and other cities with PBSPs who wish to evaluate and address spatial
inequities in accessibility and transportation options in urban areas.

\hypertarget{references}{%
\section*{References}\label{references}}
\addcontentsline{toc}{section}{References}

\hypertarget{refs}{}
\leavevmode\hypertarget{ref-auchinclossDesignBaselineDescription2020}{}%
Auchincloss, A.H., Michael, Y.L., Fuller, D., Li, S., Niamatullah, S.,
Fillmore, C.E., Setubal, C., Bettigole, C., 2020. Design and baseline
description of a cohort of bikeshare users in the city of Philadelphia.
Journal of Transport \& Health 16, 100836.
doi:\href{https://doi.org/10.1016/j.jth.2020.100836}{10.1016/j.jth.2020.100836}

\leavevmode\hypertarget{ref-babagoliExploringHealthSpatial2019}{}%
Babagoli, M.A., Kaufman, T.K., Noyes, P., Sheffield, P.E., 2019.
Exploring the health and spatial equity implications of the New York
City Bike share system. Journal of Transport \& Health 13, 200--209.
doi:\href{https://doi.org/10.1016/j.jth.2019.04.003}{10.1016/j.jth.2019.04.003}

\leavevmode\hypertarget{ref-bivand2020progress}{}%
Bivand, R.S., 2020. Progress in the r ecosystem for representing and
handling spatial data. Journal of Geographical Systems 1--32.
doi:\href{https://doi.org/10.1007/s10109-020-00336-0}{10.1007/s10109-020-00336-0}

\leavevmode\hypertarget{ref-breyWantRideMy2017}{}%
Brey, R., Castillo-Manzano, J.I., Castro-Nuño, M., 2017. ``I want to
ride my bicycle'': Delimiting cyclist typologies. Applied Economics
Letters 24, 549--552.
doi:\href{https://doi.org/10.1080/13504851.2016.1210760}{10.1080/13504851.2016.1210760}

\leavevmode\hypertarget{ref-brunsdon2020opening}{}%
Brunsdon, C., Comber, A., 2020. Opening practice: Supporting
reproducibility and critical spatial data science. Journal of
Geographical Systems 1--20.
doi:\href{https://doi.org/10.1007/s10109-020-00334-2}{10.1007/s10109-020-00334-2}

\leavevmode\hypertarget{ref-buckAreBikeshareUsers2013}{}%
Buck, D., Buehler, R., Happ, P., Rawls, B., Chung, P., Borecki, N.,
2013. Are Bikeshare Users Different from Regular Cyclists?: A First Look
at Short-Term Users, Annual Members, and Area Cyclists in the
Washington, D.C., Region. Transportation Research Record 2387, 112--119.
doi:\href{https://doi.org/10.3141/2387-13}{10.3141/2387-13}

\leavevmode\hypertarget{ref-caspiBikesharingPhiladelphiaLowerincome2019}{}%
Caspi, O., Noland, R.B., 2019. Bikesharing in Philadelphia: Do
lower-income areas generate trips? Travel Behaviour and Society 16,
143--152.
doi:\href{https://doi.org/10.1016/j.tbs.2019.05.004}{10.1016/j.tbs.2019.05.004}

\leavevmode\hypertarget{ref-chenExploringEquityPerformance2019}{}%
Chen, Z., Guo, Y., Stuart, A.L., Zhang, Y., Li, X., 2019. Exploring the
equity performance of bike-sharing systems with disaggregated data: A
story of southern Tampa. Transportation Research Part A: Policy and
Practice 130, 529--545.
doi:\href{https://doi.org/10.1016/j.tra.2019.09.048}{10.1016/j.tra.2019.09.048}

\leavevmode\hypertarget{ref-chenUnobservedHeterogeneityTransportation2021}{}%
Chen, Z., Li, X., 2021. Unobserved heterogeneity in transportation
equity analysis: Evidence from a bike-sharing system in southern Tampa.
Journal of Transport Geography 91, 102956.
doi:\href{https://doi.org/10.1016/j.jtrangeo.2021.102956}{10.1016/j.jtrangeo.2021.102956}

\leavevmode\hypertarget{ref-civicplanSoBiHamiltonMembership2017}{}%
Civicplan, 2017. SoBi Hamilton Membership Survey. Civicplan \textbar{}
Planning Engagement Research.

\leavevmode\hypertarget{ref-fishmanBikeshareReviewRecent2016}{}%
Fishman, E., 2016. Bikeshare: A Review of Recent Literature. Transport
Reviews 36, 92--113.
doi:\href{https://doi.org/10.1080/01441647.2015.1033036}{10.1080/01441647.2015.1033036}

\leavevmode\hypertarget{ref-fishmanBarriersBikesharingAnalysis2014}{}%
Fishman, E., Washington, S., Haworth, N., Mazzei, A., 2014. Barriers to
bikesharing: An analysis from Melbourne and Brisbane. Journal of
Transport Geography 41, 325--337.
doi:\href{https://doi.org/10.1016/j.jtrangeo.2014.08.005}{10.1016/j.jtrangeo.2014.08.005}

\leavevmode\hypertarget{ref-fullerUseNewPublic2011}{}%
Fuller, D., Gauvin, L., Kestens, Y., Daniel, M., Fournier, M., Morency,
P., Drouin, L., 2011. Use of a New Public Bicycle Share Program in
Montreal, Canada. American Journal of Preventive Medicine 41, 80--83.
doi:\href{https://doi.org/10.1016/j.amepre.2011.03.002}{10.1016/j.amepre.2011.03.002}

\leavevmode\hypertarget{ref-geursAccessibilityEvaluationLanduse2004}{}%
Geurs, K.T., van Wee, B., 2004. Accessibility evaluation of land-use and
transport strategies: Review and research directions. Journal of
Transport Geography 12, 127--140.
doi:\href{https://doi.org/10.1016/j.jtrangeo.2003.10.005}{10.1016/j.jtrangeo.2003.10.005}

\leavevmode\hypertarget{ref-hamiltonHamiltonBikeShare2015}{}%
Hamilton, C. of, 2015. Hamilton Bike Share.

\leavevmode\hypertarget{ref-handyMeasuringAccessibilityExploration1997}{}%
Handy, S.L., Niemeier, D.A., 1997. Measuring Accessibility: An
Exploration of Issues and Alternatives. Environment and Planning A:
Economy and Space 29, 1175--1194.
doi:\href{https://doi.org/10.1068/a291175}{10.1068/a291175}

\leavevmode\hypertarget{ref-hosfordEvaluationImpactPublic2018}{}%
Hosford, K., Fuller, D., Lear, S.A., Teschke, K., Gauvin, L., Brauer,
M., Winters, M., 2018. Evaluation of the impact of a public bicycle
share program on population bicycling in Vancouver, BC. Preventive
Medicine Reports 12, 176--181.
doi:\href{https://doi.org/10.1016/j.pmedr.2018.09.014}{10.1016/j.pmedr.2018.09.014}

\leavevmode\hypertarget{ref-hosfordWhoArePublic2018}{}%
Hosford, K., Winters, M., 2018. Who Are Public Bicycle Share Programs
Serving? An Evaluation of the Equity of Spatial Access to Bicycle Share
Service Areas in Canadian Cities. Transportation Research Record 2672,
42--50.
doi:\href{https://doi.org/10.1177/0361198118783107}{10.1177/0361198118783107}

\leavevmode\hypertarget{ref-hosfordEvaluatingImpactImplementing2019}{}%
Hosford, K., Winters, M., Gauvin, L., Camden, A., Dubé, A.-S., Friedman,
S.M., Fuller, D., 2019. Evaluating the impact of implementing public
bicycle share programs on cycling: The International Bikeshare Impacts
on Cycling and Collisions Study (IBICCS). International Journal of
Behavioral Nutrition and Physical Activity 16, 107.
doi:\href{https://doi.org/10.1186/s12966-019-0871-9}{10.1186/s12966-019-0871-9}

\leavevmode\hypertarget{ref-hullgrassoBikeShareEquity2020}{}%
Hull Grasso, S., Barnes, P., Chavis, C., 2020. Bike Share Equity for
Underrepresented Groups: Analyzing Barriers to System Usage in
Baltimore, Maryland. Sustainability 12, 7600.
doi:\href{https://doi.org/10.3390/su12187600}{10.3390/su12187600}

\leavevmode\hypertarget{ref-kabraBikeShareSystemsAccessibility2020}{}%
Kabra, A., Belavina, E., Girotra, K., 2020. Bike-Share Systems:
Accessibility and Availability. Management Science 66, 3803--3824.
doi:\href{https://doi.org/10.1287/mnsc.2019.3407}{10.1287/mnsc.2019.3407}

\leavevmode\hypertarget{ref-kwanSpaceTimeIntegral1998}{}%
Kwan, M.-P., 1998. Space-Time and Integral Measures of Individual
Accessibility: A Comparative Analysis Using a Point-based Framework.
Geographical Analysis 30, 191--216.
doi:\href{https://doi.org/10.1111/j.1538-4632.1998.tb00396.x}{10.1111/j.1538-4632.1998.tb00396.x}

\leavevmode\hypertarget{ref-larsenQuarterMileReexamining2010}{}%
Larsen, J., El-Geneidy, A., Yasmin, F., 2010. Beyond the Quarter Mile:
Re-examining Travel Distances by Active Transportation. Canadian Journal
of Urban Research 19, 70--88.

\leavevmode\hypertarget{ref-levineCenturyEvolutionAccessibility2020}{}%
Levine, J., 2020. A century of evolution of the accessibility concept.
Transportation Research Part D: Transport and Environment 83, 102309.
doi:\href{https://doi.org/10.1016/j.trd.2020.102309}{10.1016/j.trd.2020.102309}

\leavevmode\hypertarget{ref-lovelace2021open}{}%
Lovelace, R., 2021. Open source tools for geographic analysis in
transport planning. Journal of Geographical Systems 1--32.
doi:\href{https://doi.org/doi.org/10.1007/s10109-020-00342-2}{doi.org/10.1007/s10109-020-00342-2}

\leavevmode\hypertarget{ref-luUnderstandingBikeShare2018}{}%
Lu, W., Scott, D.M., Dalumpines, R., 2018. Understanding bike share
cyclist route choice using GPS data: Comparing dominant routes and
shortest paths. Journal of Transport Geography 71, 172--181.
doi:\href{https://doi.org/10.1016/j.jtrangeo.2018.07.012}{10.1016/j.jtrangeo.2018.07.012}

\leavevmode\hypertarget{ref-macarthurAdaptiveBikeShare2020}{}%
MacArthur, J., McNeil, N., Cummings, A., Broach, J., 2020. Adaptive Bike
Share: Expanding Bike Share to People with Disabilities and Older
Adults. Transportation Research Record 2674, 556--565.
doi:\href{https://doi.org/10.1177/0361198120925079}{10.1177/0361198120925079}

\leavevmode\hypertarget{ref-millwardActivetransportWalkingBehavior2013}{}%
Millward, H., Spinney, J., Scott, D., 2013. Active-transport walking
behavior: Destinations, durations, distances. Journal of Transport
Geography 28, 101--110.
doi:\href{https://doi.org/10.1016/j.jtrangeo.2012.11.012}{10.1016/j.jtrangeo.2012.11.012}

\leavevmode\hypertarget{ref-mooneyFreedomStationSpatial2019}{}%
Mooney, S.J., Hosford, K., Howe, B., Yan, A., Winters, M., Bassok, A.,
Hirsch, J.A., 2019. Freedom from the station: Spatial equity in access
to dockless bike share. Journal of Transport Geography 74, 91--96.
doi:\href{https://doi.org/10.1016/j.jtrangeo.2018.11.009}{10.1016/j.jtrangeo.2018.11.009}

\leavevmode\hypertarget{ref-nickkarSpatialtemporalGenderLand2019}{}%
Nickkar, A., Banerjee, S., Chavis, C., Bhuyan, I.A., Barnes, P., 2019. A
spatial-temporal gender and land use analysis of bikeshare ridership:
The case study of Baltimore City. City, Culture and Society 18, 100291.
doi:\href{https://doi.org/10.1016/j.ccs.2019.100291}{10.1016/j.ccs.2019.100291}

\leavevmode\hypertarget{ref-paezDemandLevelService2019}{}%
Paez, A., Higgins, C.D., Vivona, S.F., 2019. Demand and level of service
inflation in Floating Catchment Area (FCA) methods. PLoS ONE 14.
doi:\href{https://doi.org/10.1371/journal.pone.0218773}{10.1371/journal.pone.0218773}

\leavevmode\hypertarget{ref-paezMeasuringAccessibilityPositive2012}{}%
Páez, A., Scott, D.M., Morency, C., 2012. Measuring accessibility:
Positive and normative implementations of various accessibility
indicators. Journal of Transport Geography, Special Section on
Accessibility and Socio-Economic Activities: Methodological and
Empirical Aspects 25, 141--153.
doi:\href{https://doi.org/10.1016/j.jtrangeo.2012.03.016}{10.1016/j.jtrangeo.2012.03.016}

\leavevmode\hypertarget{ref-pereiraGeographicAccessCOVID192021}{}%
Pereira, R.H.M., Braga, C.K.V., Servo, L.M., Serra, B., Amaral, P.,
Gouveia, N., Paez, A., 2021. Geographic access to COVID-19 healthcare in
Brazil using a balanced float catchment area approach. Social Science \&
Medicine 273, 113773.
doi:\href{https://doi.org/10.1016/j.socscimed.2021.113773}{10.1016/j.socscimed.2021.113773}

\leavevmode\hypertarget{ref-Pereira2021r5r}{}%
Pereira, R.H.M., Saraiva, M., Herszenhut, D., Braga, C.K.V., Conway,
M.W., 2021. R5r: Rapid realistic routing on multimodal transport
networks with r\textsuperscript{5} in r. Findings.
doi:\href{https://doi.org/10.32866/001c.21262}{10.32866/001c.21262}

\leavevmode\hypertarget{ref-qianBikesharingEquityDisadvantaged2020}{}%
Qian, X., Jaller, M., 2020. Bikesharing, equity, and disadvantaged
communities: A case study in Chicago. Transportation Research Part A:
Policy and Practice 140, 354--371.
doi:\href{https://doi.org/10.1016/j.tra.2020.07.004}{10.1016/j.tra.2020.07.004}

\leavevmode\hypertarget{ref-qianEnhancingEquitableService2020}{}%
Qian, X., Jaller, M., Niemeier, D., 2020. Enhancing equitable service
level: Which can address better, dockless or dock-based Bikeshare
systems? Journal of Transport Geography 86, 102784.
doi:\href{https://doi.org/10.1016/j.jtrangeo.2020.102784}{10.1016/j.jtrangeo.2020.102784}

\leavevmode\hypertarget{ref-reillyNoncyclistsFrequentCyclists2020}{}%
Reilly, K.H., Noyes, P., Crossa, A., 2020. From non-cyclists to frequent
cyclists: Factors associated with frequent bike share use in New York
City. Journal of Transport \& Health 16, 100790.
doi:\href{https://doi.org/10.1016/j.jth.2019.100790}{10.1016/j.jth.2019.100790}

\leavevmode\hypertarget{ref-reillyGenderDisparitiesNew2020}{}%
Reilly, K.H., Wang, S.M., Crossa, A., 2020. Gender disparities in New
York City bike share usage. International Journal of Sustainable
Transportation 0, 1--9.
doi:\href{https://doi.org/10.1080/15568318.2020.1861393}{10.1080/15568318.2020.1861393}

\leavevmode\hypertarget{ref-scottWhatFactorsInfluence2019}{}%
Scott, D.M., Ciuro, C., 2019. What factors influence bike share
ridership? An investigation of Hamilton, Ontario's bike share hubs.
Travel Behaviour and Society 16, 50--58.
doi:\href{https://doi.org/10.1016/j.tbs.2019.04.003}{10.1016/j.tbs.2019.04.003}

\leavevmode\hypertarget{ref-scottRouteChoiceBike2021}{}%
Scott, D.M., Lu, W., Brown, M.J., 2021. Route choice of bike share
users: Leveraging GPS data to derive choice sets. Journal of Transport
Geography 90, 102903.
doi:\href{https://doi.org/10.1016/j.jtrangeo.2020.102903}{10.1016/j.jtrangeo.2020.102903}

\leavevmode\hypertarget{ref-smith2015exploring}{}%
Smith, C.S., Oh, J.-S., Lei, C., 2015. Exploring the equity dimensions
of us bicycle sharing systems. Western Michigan University.
Transportation Research Center for Livable~\ldots.

\leavevmode\hypertarget{ref-toblerSmoothPycnophylacticInterpolation1979}{}%
Tobler, W.R., 1979. Smooth Pycnophylactic Interpolation for Geographical
Regions. Journal of the American Statistical Association 74, 519--530.
doi:\href{https://doi.org/10.1080/01621459.1979.10481647}{10.1080/01621459.1979.10481647}

\leavevmode\hypertarget{ref-wangNeighborhoodSociodemographicCharacteristics2019}{}%
Wang, J., Lindsey, G., 2019a. Neighborhood socio-demographic
characteristics and bike share member patterns of use. Journal of
Transport Geography 79, 102475.
doi:\href{https://doi.org/10.1016/j.jtrangeo.2019.102475}{10.1016/j.jtrangeo.2019.102475}

\leavevmode\hypertarget{ref-wangNewBikeShare2019}{}%
Wang, J., Lindsey, G., 2019b. Do new bike share stations increase member
use: A quasi-experimental study. Transportation Research Part A: Policy
and Practice 121, 1--11.
doi:\href{https://doi.org/10.1016/j.tra.2019.01.004}{10.1016/j.tra.2019.01.004}

\leavevmode\hypertarget{ref-wintersWhoAreSuperusers2019}{}%
Winters, M., Hosford, K., Javaheri, S., 2019. Who are the
``super-users'' of public bike share? An analysis of public bike share
members in Vancouver, BC. Preventive Medicine Reports 15, 100946.
doi:\href{https://doi.org/10.1016/j.pmedr.2019.100946}{10.1016/j.pmedr.2019.100946}


\end{document}


